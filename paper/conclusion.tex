%
%---------------------------------------------------------
% Header: This file includes all the packages and custom definitions
% we have used in this example thesis.
%----------------------------------------------------------

\documentclass[12pt,twoside,openright]{report}

%---------------------
% START: Packages
%---------------------
\usepackage{textcomp}
\usepackage[latin1]{inputenc}
\usepackage{amsmath}
\usepackage{amsfonts}
\usepackage{amssymb}
\usepackage{amsthm}
\usepackage{graphicx}
\usepackage{soul}
\usepackage{listings}
%\usepackage{subfig}
\usepackage{verbatim}
\usepackage{alltt}
\usepackage{siunitx}
\usepackage{color}

\usepackage{caption}
\usepackage{subcaption}
\usepackage{lmodern}

% There are used in the graphics chapter.
% You can delete the following two lines if you use no tikz/pgf graphics
% in your thesis.
\usepackage{tikz}
\usepackage{pgfplots}

% Load the natbib citation package: set the citations to be numerical
% with square brackets separated by commas.
\usepackage[numbers,square,comma]{natbib}

% Now include the Acadia thesis style
\usepackage{acadia-hon-thesis}

% Load the hyperref package.
% The options tell it to (a) use hyper-links to pages with Roman
% numerals that are different than pages with Arabic numbers, and
% (b) tell Adobe reader to show a page number matching the thesis page
% number (rather than sequentially numbering the PDF pages from 1).
\usepackage[plainpages=false,pdfpagelabels]{hyperref}

% The information in the first three lines here goes into the PDF
% document properties.
% The rest of the lines define options related to hyper-links.
% colorlinks: typeset links in the given colours
%	      (otherwise an ugly box is drawn around the links, although
%	       it is only seen on the screen, not in printed copies)
% A newer option (since May 2011) would be to just use the hidelinks option.
% Note: pdfprintscaling=None should discourage Adobe reader from wanting to
% scale your pages to fit printable area when you print from Adobe reader.
\hypersetup{%
    pdftitle={Digital In-line Holography and the Numerical Reconstruction of Holograms},
    pdfauthor={Wesley A. Bowman},
    pdfkeywords={holography,microscopy,in-line,hologram},
    colorlinks = true,
    linkcolor = black,
    anchorcolor = black,
    citecolor = black,
    filecolor = black,
    urlcolor = black,
    pdfprintscaling=None
}


% Load the algorithm/mic packages and use chapter-wise numbering
\usepackage[chapter]{algorithm}
\usepackage{algorithmic}



%---------------------
% END: Packages
%---------------------
\bibliographystyle{plainnat}

% The depth of the table of contents: change the MAXIMUM depth of
% citations in your table of contents.
\setcounter{tocdepth}{6}

%
% Some definitions of commands used in this thesis
%

% For instance, if you have an acronym you like to use, then define a
% command, it's faster and if the acronym changes you only have to
% change it in one place.
\def\sysacro{SPECIALACRONYM}

% Allow us to change the margins easily and at will
\newenvironment{changemargin}[2]{%
  \begin{list}{}{%
    \setlength{\topsep}{0pt}%
    \setlength{\leftmargin}{#1}%
    \setlength{\rightmargin}{#2}%
    \setlength{\listparindent}{\parindent}%
    \setlength{\itemindent}{\parindent}%
    \setlength{\parsep}{\parskip}%
  }%
  \item[]}{\end{list}}

%setup the default format of listings
\lstset{%
    basicstyle=\footnotesize,
    numbers=left,
    xleftmargin=5mm,
    linewidth=\textwidth,
    breaklines,
    frame=tb,
    frameround=fttt
}

\definecolor{dkgreen}{rgb}{0,0.6,0}
\definecolor{gray}{rgb}{0.5,0.5,0.5}
\definecolor{mauve}{rgb}{0.58,0,0.82}

% A new definition style
\newtheoremstyle{defstyle}	% name
    {3pt}			% Space above
    {3pt}			% Space below
    {}				% Body font
    {}				% Indent amount
    {\itshape}			% Theorem head font
    {:}				% Punctuation after theorem head
    {.5em}			% Space after theorem head
    {}		% Theorem head spec (can be left empty,meaning 'normal�)
\theoremstyle{definition}
\newtheorem{definition}{Definition}[chapter]


% Change comment style for algorithms
\renewcommand{\algorithmiccomment}[1]{/*#1*/}
% Change Require: to Input: for algorithms
\renewcommand{\algorithmicrequire}{\textbf{Input:}}
% Change Ensure: to Output: for algorithms
\renewcommand{\algorithmicensure}{\textbf{Output:}}

\newcommand{\BigO}[1]{\ensuremath{\operatorname{O}\bigl(#1\bigr)}}

\newfloat{Code}{H}{myc}

%\begin{document}
\chapter{Conclusions and Recommendations} \label{chap:Conclusion}

A digital in-line holographic microscope was built at Acadia University
so that holograms could be recorded and reconstructed numerically 
for use in research and course work.

The Harvard University setup was much more sophisticated and expensive which
allowed for finer adjustments to be made to the microscope. This helped the
reconstruction process since it was known at which $z$-values the objects where
away from the focal length of the objectives. The Harvard setup also allowed
a quick succession of images to be recorded leading to 3D holographic
reconstructions to
be made over a short time frame.
With the
setup that was built at Acadia University, there were
reconstructions made that could resolve 5 micron colloidal diamond up to a
lateral separation of \SI{2.52}{\micro\meter}. Biological
samples, such as red blood cells or onion cells, were imaged, but due to the
high number of
fringes and overlapping objects, they could not all be resolved 
and hence satisfactorily reconstructed.
%to the standard
%that was being sought. 
Overall, the images and reconstructions obtained
contained small details of a few microns in size that could be resolved,
and the objective of this experiment was achieved.

Future work that could be done would be to upgrade the microscope components.
Specifically, upgrading the second objective lens so that it has a higher magnification
would allow for smaller objects to be observed since it would produce more fringes and
allow for a higher resolution reconstruction to be achieved. This would help
with biological samples that are around 1 micron
in size, and the objects in the slides that are close to one another. This higher
magnification would allow for less interference between these close objects,
also leading to better resolution of the objects. Another improvement to be
made to the setup would be to take the horizontal positioning and turn it into a
vertical setup. This would allow for a more stable suspension on the slide
by reducing the gravitational effects on the samples. This would stop the
objects in the solutions from trending downwards toward the bottom of the well, allowing
for a longer period to take the holograms.
%The objects could still be
%made to move if that was what was desired, but it would not be mandatory that
%they move since the gravitational effects would be countered.

%With a
%second objective that had a higher magnification, the reconstructions
%could have a better resolution,
%allowing for some of the biological samples that could not be resolved to be
%reconstructed successfully. Another improvement would be to translate the
%horizontal setup of the microscope into a vertical setup. This would allow for
%a better suspension to be had on the slide by reducing the gravitational
%effects on the samples. The horizontal method makes any
%suspension slowly decay to the bottom of the well, when it would be
%advantageous for the objects to not move unless it was desired for them to
%move.


%\end{document}
