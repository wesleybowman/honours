%---------------------------------------------------------
% Preliminaries: Set up your own details in this file!
%----------------------------------------------------------


\title{DHM (Don't Harm Me)}
\author{Wesley A. Bowman}
\dept{Physics}   % E.g., Physics, Computer Science,
\deptOrSchool{Department}   % OR MAYBE School (one or the other)
\degree{Science}
\submissionMonth{March}	    % OR WHATEVER MONTH YOU ACTUALLY SUBMIT IN
\submissionYear{YYYY}
\copyrightYear{YYYY}
% Use a "~" after the "r." of "Dr." so that TeX doesn't think you have
% ended a sentence (at which point it gives extra space).
\supervisor{Dr.~Michael Robertson}
% Remove the '%' from the next linebreake and fill in the name if desired.
\cosupervisor{Dr.~Peter Williams}
\headOrDirector{Dr.~Craig Bennett}

\beforepreface
\prefacesection{Abstract}

This is a ``quick-start'' guide to help you use \LaTeX\ to produce
your thesis.  It by no means covers all issues, but it should give you
a solid start.

\TeX\ is a system developed by Donald Knuth with the goal of doing
beautiful typesetting, particularly for documents which use
mathematics.  It is one of the first significant programs made
available to the world at large (for free) by its author.  By design,
Knuth has dictated that \TeX\ won't change any more, which means
that \TeX\ documents won't become unusable because of incompatible
updates.  (However, other people are extending \TeX, so it is not a
``dead'' system.)

\LaTeX\ is a set of additions to \TeX\ which arguably make it easier
to produce documents.  This sample thesis will assume you are going to
use \LaTeX\, but if you later to decide to try Knuth's ``plain'' \TeX,
you will find that the concepts are similar, but some details are
different.

This document was mostly written by Brian Demmings; any ``I'' found in
this document refers to Brian.  Jim Diamond has updated it
occasionally since Brian graduated, and hopes he has only improved
things with his changes.  Kudos to Alex Sanford for modifying the
master's thesis template to meet the honours thesis requirements.

The ``source code'' of this document should be in the same directory
as the nicely-formatted PDF file you are reading.  If you want to see
exactly how anything here is produced, just look in the appropriate
``source code'' file.  (For example\dots\ if you read the source code,
you will see that we used ``\verb|\dots|'' rather than ``\verb|...|'',
since the spacing of ``\dots'' is better than ``...''.)

\medskip

Write \emph{your} abstract here.
		
\textbf{BEWARE:} Some of the techniques (e.g., the ``Table of
Algorithms/Listings'') may \emph{not} be permitted.
When in doubt check with the registrar or your adviser.






\newpage
	
\prefacesection{Acknowledgments}

Place any acknowledgments you might want to make here.

\afterpreface
