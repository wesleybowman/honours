%---------------------------------------------------------
% Preliminaries: Set up your own details in this file!
%----------------------------------------------------------


\title{Digital In-Line Holography and the Numerical Reconstruction of Holograms}
\author{Wesley A. Bowman}
\dept{Physics}   % E.g., Physics, Computer Science,
\deptOrSchool{Department}   % OR MAYBE School (one or the other)
\degree{Science}
\submissionMonth{March}	    % OR WHATEVER MONTH YOU ACTUALLY SUBMIT IN
\submissionYear{2014}
\copyrightYear{2014}
% Use a "~" after the "r." of "Dr." so that TeX doesn't think you have
% ended a sentence (at which point it gives extra space).
\supervisor{Dr.~Michael Robertson}
% Remove the '%' from the next linebreake and fill in the name if desired.
\cosupervisor{Dr.~Peter Williams}
\headOrDirector{Dr.~Craig Bennett}
\tablespagefalse

\beforepreface
\prefacesection{Abstract}

    A hologram is the recorded interference pattern between a reference
    wave and a scattered wave from an object. Although, the recording medium only
    captures
    the intensity of the light from these waves, the information about both
    the amplitude and the phase is still present in this recording, due to the
    interference fringes produced by the superposition of these two waves.
    This information can be reconstructed numerically using the
    Fresnel-Kirchoff integral, which can be transformed into a
    numerically efficient calculation using the convolution theorem.

    The setup used in this thesis was in-line holography.
    In Digital In-line Holographic Microscopy (DIHM), all of the components 
    are aligned in a linear setup, and 
    both the reference and scattered waves are
    recorded on the same detector. The setup used in this thesis contained a
    laser, two objective lenses, the object holder, a neutral density (ND) filter, and
    a charged coupled device (CCD). The CCD allowed for direct 
    digital recording of the holograms.

    Once the holograms were obtained from the DIHM, an open-source Python
    library, HoloPy, was used to reconstruct the holographic images. Images
    from both Harvard and Acadia Universities were recorded and reconstructed, which
    demonstrated the successful
    acquisition of holograms and the reconstruction method.
    The DIHM that was constructed at Acadia achieved a resolution of approximately
    \SI{2.7}{\micro\meter}.





\newpage
	
\prefacesection{Acknowledgments}

    First and foremost, I would like to think Dr.\ Michael Robertson. Without him, 
    none of this would have been possible for numerious reasons. He gave me, not 
    only the opportunity to work with him, but also the guidance I needed to make 
    sure that this thesis could be completed. I could not have asked for a better
    supervisor. He went above and beyond, which allowed me to experience the best parts
    of the scientific community. Not only was he a mentor, he was also a great friend.
    Thank you Dr.\ Robertson.

    Next, I would like to acknowledge my family. Without them, I would never have come to
    Acadia in the first place. They showed immense support with my decision, and offered
    nothing but help with the transition. Specifically to Dad and Mom, thank you both for
    dealing with my lack of communication. A honours undergraduate degree is more demanding
    than I can ever explain, but that is no excuse. Thank you for being there when I needed
    you to be. I love you both with all of my heart. Never forget that.

    To my colleges and companions, thank you for making this a fun ride. Everyone that is
    graduating with me, thank you, cause without you I wouldn't be here.
    Robie Hennigar, thank you for the summer work we did together. You helped me troubleshoot
    and learn new skills that I will take forward with me, and I hope we never lose contact.
    Patrick Murray, thank you for being the friend I needed, when I needed him the most. You
    will always be remembered, and always be my friend.
    Aidan Bharath, I have spent way too much time with you, but you've made this last year
    the best it could have been. We've shared line after line of code, and it was worth 
    every second. You have taught me a ton of life skills, and shared more stories than
    I could, or would, ever retell. Thank you friend, for being there.

    Last, and surely not the least, Lydia van Vilsteren. I can honestly say that without
    you, none of this would have been possible. You deal with me when I am at my most
    unreasonable, you help take care of all the things that I just don't want to do,
    you accept me for who I am,
    and you remind me every day of what I am working to achieve. You have kept me on task,
    you have kept me grounded,
    you have been my stress relief, and you have just been there for me, which is more than
    a man could ever ask for. It is hard to put down just how much you mean to me.
    I am one lucky guy. Thank you for being you, and for being 
    there for me. I love you.

    And I love you all. Thank you, from my heart, as you will all have a place
    in it.

\afterpreface
