%
%---------------------------------------------------------
% Header: This file includes all the packages and custom definitions
% we have used in this example thesis.
%----------------------------------------------------------

\documentclass[12pt,twoside,openright]{report}

%---------------------
% START: Packages
%---------------------
\usepackage{textcomp}
\usepackage[latin1]{inputenc}
\usepackage{amsmath}
\usepackage{amsfonts}
\usepackage{amssymb}
\usepackage{amsthm}
\usepackage{graphicx}
\usepackage{soul}
\usepackage{listings}
%\usepackage{subfig}
\usepackage{verbatim}
\usepackage{alltt}
\usepackage{siunitx}
\usepackage{color}

\usepackage{caption}
\usepackage{subcaption}
\usepackage{lmodern}

% There are used in the graphics chapter.
% You can delete the following two lines if you use no tikz/pgf graphics
% in your thesis.
\usepackage{tikz}
\usepackage{pgfplots}

% Load the natbib citation package: set the citations to be numerical
% with square brackets separated by commas.
\usepackage[numbers,square,comma]{natbib}

% Now include the Acadia thesis style
\usepackage{acadia-hon-thesis}

% Load the hyperref package.
% The options tell it to (a) use hyper-links to pages with Roman
% numerals that are different than pages with Arabic numbers, and
% (b) tell Adobe reader to show a page number matching the thesis page
% number (rather than sequentially numbering the PDF pages from 1).
\usepackage[plainpages=false,pdfpagelabels]{hyperref}

% The information in the first three lines here goes into the PDF
% document properties.
% The rest of the lines define options related to hyper-links.
% colorlinks: typeset links in the given colours
%	      (otherwise an ugly box is drawn around the links, although
%	       it is only seen on the screen, not in printed copies)
% A newer option (since May 2011) would be to just use the hidelinks option.
% Note: pdfprintscaling=None should discourage Adobe reader from wanting to
% scale your pages to fit printable area when you print from Adobe reader.
\hypersetup{%
    pdftitle={Digital In-line Holography and the Numerical Reconstruction of Holograms},
    pdfauthor={Wesley A. Bowman},
    pdfkeywords={holography,microscopy,in-line,hologram},
    colorlinks = true,
    linkcolor = black,
    anchorcolor = black,
    citecolor = black,
    filecolor = black,
    urlcolor = black,
    pdfprintscaling=None
}


% Load the algorithm/mic packages and use chapter-wise numbering
\usepackage[chapter]{algorithm}
\usepackage{algorithmic}



%---------------------
% END: Packages
%---------------------
\bibliographystyle{plainnat}

% The depth of the table of contents: change the MAXIMUM depth of
% citations in your table of contents.
\setcounter{tocdepth}{6}

%
% Some definitions of commands used in this thesis
%

% For instance, if you have an acronym you like to use, then define a
% command, it's faster and if the acronym changes you only have to
% change it in one place.
\def\sysacro{SPECIALACRONYM}

% Allow us to change the margins easily and at will
\newenvironment{changemargin}[2]{%
  \begin{list}{}{%
    \setlength{\topsep}{0pt}%
    \setlength{\leftmargin}{#1}%
    \setlength{\rightmargin}{#2}%
    \setlength{\listparindent}{\parindent}%
    \setlength{\itemindent}{\parindent}%
    \setlength{\parsep}{\parskip}%
  }%
  \item[]}{\end{list}}

%setup the default format of listings
\lstset{%
    basicstyle=\footnotesize,
    numbers=left,
    xleftmargin=5mm,
    linewidth=\textwidth,
    breaklines,
    frame=tb,
    frameround=fttt
}

\definecolor{dkgreen}{rgb}{0,0.6,0}
\definecolor{gray}{rgb}{0.5,0.5,0.5}
\definecolor{mauve}{rgb}{0.58,0,0.82}

% A new definition style
\newtheoremstyle{defstyle}	% name
    {3pt}			% Space above
    {3pt}			% Space below
    {}				% Body font
    {}				% Indent amount
    {\itshape}			% Theorem head font
    {:}				% Punctuation after theorem head
    {.5em}			% Space after theorem head
    {}		% Theorem head spec (can be left empty,meaning 'normal�)
\theoremstyle{definition}
\newtheorem{definition}{Definition}[chapter]


% Change comment style for algorithms
\renewcommand{\algorithmiccomment}[1]{/*#1*/}
% Change Require: to Input: for algorithms
\renewcommand{\algorithmicrequire}{\textbf{Input:}}
% Change Ensure: to Output: for algorithms
\renewcommand{\algorithmicensure}{\textbf{Output:}}

\newcommand{\BigO}[1]{\ensuremath{\operatorname{O}\bigl(#1\bigr)}}

\newfloat{Code}{H}{myc}

%\begin{document}

\chapter{Results and Discussion} \label{chap:Results}

Two sets of experimental holograms were obtained at Harvard and Acadia
Universities and are reported in this chapter. The main goal was to achieve
high resolution 
reconstructions that correctly represented the objects in the holographic
images.
%holograms that correctly reconstructed the objects in the
%holographic images. 
The holographic images taken at Harvard University were
used as a reference to ensure that both the images obtained from Acadia
University and the reconstruction procedure were correct.

\section{Harvard University}

As mentioned previously, HoloPy was the open-source Python module used for all
of the image reconstructions
done, and was developed by the Manoharan Lab, a world leader in optical
holography research \cite{Manoharan}. 
Through discussions arising from the use of HoloPy with the software developer, 
an invitation to travel to
Harvard University and meet with the developer to obtain some insight on how to
build an affective holographic microscope, as well as to learn how to properly
use HoloPy was offered.
All of the reconstructions calculated from the Harvard University images had a pixel
spacing of 0.17 microns and the image size was 1024 by 1024 pixels.
Therefore, the $x$ and $y$ dimensions of each image is \SI{174.08}{\micro\meter} by \SI{174.08}{\micro\meter}.
%
%When this module was found, the opportunity was presented to travel to
%Harvard University and meet with the developer to get some incite on how to
%build an efficient microscope as well as to learn how to properly use HoloPy. 
%

While at
Harvard University, the Manoharan Lab graciously allowed for holograms to be
recorded using their setup so that there would be a reference set of images to
compare to the microscope
that was to be built at Acadia. The Manoharan Lab optical holography setup is shown in
Figure~\ref{fig:harvardSetup}. This was a highly advanced setup allowing for
fine adjustments to be made easily to the any part of the setup. In this setup,
the samples were mounted horizontally to minimize the settling of the specimens
due to gravitational effects.



\begin{figure}[ht!]
    \begin{center}

        \begin{subfigure}[t]{0.4\textwidth}
            \label{fig:second}
            \includegraphics[width=\textwidth]{pictures/20130814_141000.jpg}
            \caption{The Harvard setup had both bright field and holographic
            imaging capabilities.}
        \end{subfigure}
            %
                \hspace*{\fill}
    %
        \begin{subfigure}[t]{0.4\textwidth}
            \label{fig:third}
            \includegraphics[width=\textwidth]{pictures/20130814_141007.jpg}
            \caption{The monitor allowed for real time viewing of the images
            before and after acquisition.}
        \end{subfigure}
        \\

        \begin{subfigure}[t]{0.4\textwidth}
            \label{fig:fourth}
            \includegraphics[width=\textwidth]{pictures/20130814_141151.jpg}
            \caption{The sample with the laser beam going through it.}
        \end{subfigure}
        \hspace*{\fill}
        \begin{subfigure}[t]{0.4\textwidth}%[Caption of Second Figure]{%
            \label{fig:fifth}
            \includegraphics[width=\textwidth]{pictures/20130814_141310.jpg}
            \caption{The laser setup before it gets to the sample.}
        \end{subfigure}

    \end{center}
    \caption{%
        The Harvard University optical holography setup.
    }%
    \label{fig:harvardSetup}
\end{figure}

%While at
%Harvard University, the Manoharan Lab graciously allowed for holograms to be
%recorded using their setup so that there would be a reference set of images to
%compare to the microscope
%that was to be built at Acadia. The Manoharan Lab optical holography setup is shown in
%Figure~\ref{fig:harvardSetup}. This was a highly advanced setup allowing for
%fine adjustments to be made easily to the any part of the setup. In this setup,
%the samples were mounted horizontally to minimize the settling of the specimens
%due to gravitational effects.

A lot of the images taken at Harvard University 
%had objects in them 
contained objects which were
unknown since the sample used was water from the Charles River in Boston,
Massachusetts. Not
having known what the objects were did not cause an issue since
a demonstration of holographic principles was the only purpose of looking at the objects.


\begin{figure}
    \begin{center}
        \includegraphics[width=\textwidth]{pictures/128focus.png}
        \caption{The in-focus holographic image for objects $i.$\ and  $ii.$\
            for the reconstructions
            displayed in Figures~\ref{fig:129}-\ref{fig:131}.
    Objects $i.$\ - $iii.$\ are the features being reconstructed. }
        \label{fig:128focus}
    \end{center}
\end{figure}


\begin{figure}[ht!]
    \begin{center}

            %
                %\hspace*{\fill}
    %
        \begin{subfigure}[t]{0.4\textwidth}
            \label{fig:129holo}
            \includegraphics[width=\textwidth]{pictures/129holo.png}
            \caption{A hologram of the in-focus image.}
        \end{subfigure}
        \\
        \begin{subfigure}[t]{0.4\textwidth}
            \label{fig:129rec1}
            \includegraphics[width=\textwidth]{pictures/129rec1.png}
            \caption{Reconstruction at a $z$ value of \SI{14.898}{\micro\meter}.}
        \end{subfigure}
            %
                \hspace*{\fill}
    %
%
        \begin{subfigure}[t]{0.4\textwidth}
            \label{fig:129rec2}
            \includegraphics[width=\textwidth]{pictures/129rec2.png}
            \caption{Reconstruction at a $z$ value of \SI{22.245}{\micro\meter}.}
        \end{subfigure}


    \end{center}
    \caption{%
        This hologram has the least number of fringes in this series of
        holograms at approximately 3 fringes located around object $i.$ tightly
        bound to the object. 
        Objects $i.$ and $ii.$ were reconstructed in (b),
        while object $iii.$ was reconstructed in (c).
    }%
    \label{fig:129}
\end{figure}


\begin{figure}[ht!]
    \begin{center}

        \begin{subfigure}[t]{0.4\textwidth}
            \label{fig:130holo}
            \includegraphics[width=\textwidth]{pictures/130holo.png}
            \caption{A hologram of the in-focus image.}
        \end{subfigure}
        \\
        \begin{subfigure}[t]{0.4\textwidth}
            \label{fig:130rec1}
            \includegraphics[width=\textwidth]{pictures/130rec1.png}
            \caption{Reconstruction at a $z$ value of \SI{32.245}{\micro\meter}.}
        \end{subfigure}
            %
                \hspace*{\fill}
    %
%
        \begin{subfigure}[t]{0.4\textwidth}
            \label{fig:130rec2}
            \includegraphics[width=\textwidth]{pictures/130rec2.png}
            \caption{Reconstruction at a $z$ value of \SI{39.592}{\micro\meter}.}
        \end{subfigure}


    \end{center}
    \caption{%
        This hologram is an in-between hologram in terms of the fringe density
        with there being approximately 7 tightly bound fringes around
        object $i.$
        Objects $i.$ and $ii.$ were reconstructed in (b),
        while object $iii.$ was reconstructed in (c).
    }%
    \label{fig:130}
\end{figure}


\begin{figure}[ht!]
    \begin{center}

        \begin{subfigure}[t]{0.4\textwidth}
            \label{fig:131holo}
            \includegraphics[width=\textwidth]{pictures/131holo.png}
            \caption{A hologram of the in-focus image.}
        \end{subfigure}
        \\
        \begin{subfigure}[t]{0.9\textwidth}
            \label{fig:131rec}
            \includegraphics[width=\textwidth]{pictures/131rec.png}
            \caption{Reconstruction at a $z$ value of \SI{168.367}{\micro\meter}.}
        \end{subfigure}

    \end{center}
    \caption{%
        This hologram has the most fringes with approximately 7 fringes
        grouped around object $i.$, and seemed to contain the least
        amount of detail when reconstructed. It was seen that more detail
        is shown in objects $i.$ and $ii.$, while the details of the objects at $iii.$
        were lost.
    }%
    \label{fig:131}
\end{figure}

The images shown in Figures~\ref{fig:128focus}-\ref{fig:3D} are meant to highlight a few different
examples where holographic imaging is superior to standard bright-field optical
imaging.
For Figures~\ref{fig:128focus}-\ref{fig:131}, the effect of the number of fringes in the hologram
on image reconstruction quality are presented at different image reconstruction
distances ($z$ values).
In these images,
objects $i.$ and $ii.$ were in the same focal plane, while objects at $iii.$
were in
a different focal plane. This is why multiple values of $z$ were used in
reconstruction. In Figure~\ref{fig:129} (a), there were approximately 3 fringes
located closely around object $i.$, which were tightly grouped. In
Figure~\ref{fig:130} (a), there were approximately 7 fringes
located closely around object $i.$, which were also tightly grouped. In
Figure~\ref{fig:131} (a), there were approximately 7 fringes around object
$i.$, but the fringes were not as closely bound to object $i.$ as in
Figures~\ref{fig:129} and~\ref{fig:130}. 
%This led to there being only one
%reconstruction that needed to be done since the objects at position $iii.$
%could no longer be resolved. 
Because the objects at position $iii.$ could no longer be resolved, only one
reconstruction was needed.
Fringe spacing ($\Lambda$) affects the resolution of the
hologram and is related to the wavelength and scattering angle $\theta$
%and can be shown 
by Bragg's Law,
\begin{equation}
    \Lambda = \frac{\lambda}{2\sin{\frac{\theta}{2}}} .
\end{equation}
Matt Lehmann explains the importance of interference fringes,
\begin{quote}
Every point on the recording film and on the object will provide a
slight variation to the angular relationship and consequently a variation
in fringe spacing. Since it is the fringe spacing that delineates
the object
during reconstruction, all of the optical information about the object is
recoverable and the parallax and three-dimensionality of the
reconstruction is assured.

Since the spacing of the fringes is inversely proportional to
$\sin{\frac{\theta}{2}}$, the larger the angle $\theta$ the finer the fringes.
Conversely, when the angle is small, the fringes are far apart\ldots When the
fringes are comparatively coarse, the hologram acts as a diffraction grating
and can be viewed by a point source of partially filtered white light\ldots As
the fringes become finer, the Bragg effect becomes apparent. This requires
fairly precise alignment of the hologram with the illuminating beam to effect
constructive refraction by the fringe pattern. The angle corresponds to the
conjugate of the reference angle during recording (i.e the angle the reference
beam makes with the surface of the hologram) \cite{Lehmann}.
\end{quote}
In the case that the recording medium is a CCD, the pixel size of the CCD must
be smaller than the smallest fringe spacing that needs to be recorded. Either
too few fringes, or fringes that are too close together will lead to a
reduction in resolution \cite{Jericho}. Therefore, as the fringe density gets
too high, the average number of pixels per fringes decreases which can cause
an unexpected loss of resolution. In Figures~\ref{fig:128focus}-\ref{fig:131},
this effect becomes apparent for objects $i.$ and $ii.$ where the details of the objects
seem to be more resolved as more fringes are added progressively from
Figure~\ref{fig:129} to Figure~\ref{fig:131}. However, object $iii.$ was
completely lost because their were too many fringes which resulted in a loss of
resolution.


%The advantages
%in holography are that the images do not need to be in-focus to be viewed with
%good resolution. This advantage leads to the other advantage that objects do
%not have to lie in the same plane to be viewed.

\begin{figure}[ht!]
    \begin{center}

        \begin{subfigure}[t]{0.4\textwidth}
            \label{fig:132focus}
            \includegraphics[width=\textwidth]{pictures/132focus.png}
            \caption{The in-focus holographic image.}
%                where the objects are hard to
%            identify.}
            %because of the low contrast to the background.}
        \end{subfigure}
            %
                \hspace*{\fill}
    %
        \begin{subfigure}[t]{0.4\textwidth}
            \label{fig:133holo}
            \includegraphics[width=\textwidth]{pictures/133holo.png}
            \caption{A hologram of the in-focus image.}% which already gives more
            %contrast to the in-focus image.}
        \end{subfigure}
        \\
        \begin{subfigure}[t]{\textwidth}
            \label{fig:133rec}
            \includegraphics[width=\textwidth]{pictures/133rec.png}
            \caption{Reconstruction of the hologram at a $z$ value of \SI{72.245}{\micro\meter}.}
        \end{subfigure}


    \end{center}
    \caption{%
       The reconstruction allowed for the objects $i.$ and $ii.$ to be easily
       identified since the edges of the objects now
       stand out even with the interference from the small particles around
       the objects.
%       The reconstruction allowed for the objects $i.$ and $ii.$ to be easily
%       identified because of the higher contrast. The edges of the objects now
%       stand out, even with the interference from the small particles around
%       the objects.
    }%
    \label{fig:133}
\end{figure}

A typical reconstruction is shown in Figure~\ref{fig:133}. Even with some
interference from the small objects surrounding objects $i.$ and $ii.$, the
objects were still resolved. Object $i.$ is shown to be a rectangular object
located beneath the focal plane of the small objects surrounding it. Both objects were
hard to distinguish in the in-focus holographic image, but by reconstructing
the objects, the reconstruction plane is known, and they can be
distinguished from the surrounding particles.
%into a clear
%rectangular object whose borders are clearly visible, along with object
%$ii.$, both of which were hard to distinguish in the in-focus image.


\begin{figure}
\begin{center}
    \includegraphics[width=\textwidth]{pictures/136focus.png}
    \caption{The in-focus holographic image for Figures~\ref{fig:139} and~\ref{fig:140}. }
    \label{fig:136focus}
\end{center}
\end{figure}


\begin{figure}[ht!]
    \begin{center}

        \begin{subfigure}[t]{0.4\textwidth}
            \label{fig:139holo}
            \includegraphics[width=\textwidth]{pictures/139holo.png}
            \caption{The in-focus holographic image.}
            %\caption{A hologram of the in-focus image.}
        \end{subfigure}
        \\
        \begin{subfigure}[t]{\textwidth}
            \label{fig:139rec}
            \includegraphics[width=\textwidth]{pictures/139rec.png}
            \caption{Reconstruction at a $z$ value of
                \SI{56.327}{\micro\meter}.}
        \end{subfigure}


    \end{center}
    \caption{%
        The hologram had approximately 8 fringes around the object and
        some of the details seen in the object in the in-focus can be observed.
    }%
    \label{fig:139}
\end{figure}


\begin{figure}[ht!]
    \begin{center}

        \begin{subfigure}[t]{0.4\textwidth}
            \label{fig:140holo}
            \includegraphics[width=\textwidth]{pictures/140holo.png}
            \caption{The in-focus holographic image.}
            %\caption{A hologram of the in-focus image.}
        \end{subfigure}
        \\
        \begin{subfigure}[t]{\textwidth}
            \label{fig:140rec}
            \includegraphics[width=\textwidth]{pictures/140rec.png}
            \caption{Reconstruction at a $z$ value of
                \SI{215.306}{\micro\meter}.}
        \end{subfigure}


    \end{center}
    \caption{%
        This hologram had no details observable in the in-focus image and had
        approximately 12 fringes in a wide radius around the object. The
        reconstruction showed details from the in-focus image.
    }%
    \label{fig:140}
\end{figure}

Once again, in Figures~\ref{fig:136focus}-\ref{fig:140}, the difference in
reconstruction quality with the fringe density is shown. In this case, 
more fringes led to a higher
resolution inside the object. This was because the fringe spacing was still
large enough that more fringes did indeed translate into a higher resolution.
In Figure~\ref{fig:139} there were approximately
8 fringes around the object, and in the hologram the object has some
artifacts that were observed in the in-focus holographic image
%about the object in it. 
In Figure~\ref{fig:140}, there are approximately
12-14 fringes located around the object, with the hologram looking like a white
dot and having none of the details present in Figure~\ref{fig:139}. When reconstructed,
Figure~\ref{fig:140} more closely matches the in-focus image of
Figure~\ref{fig:136focus} than Figure~\ref{fig:139} because of this higher
fringe density.

In Figure~\ref{fig:148}, the fringes in the reconstruction seem to be
interfering with one another, and this reconstruction was performed to see what
level of image resolution that could be achieved with these types of overlapping fringes.
With the use of the background-subtracted hologram a highly detailed
reconstruction was obtained, and the high contrast allowed for the small
details to be seen, which were
approximately one to two microns wide.

Since holograms contain information of the 3-D structure of the sample, a 3-D
plot of particle movement or depth profiling within the specimen can be
performed. In Figure~\ref{fig:3Dparticle}, a particle was reconstructed using over
twenty different $z$ values for a total range of 30 microns. The particle was then ploted in 3-D
space. This particle was moving when it came into view, so multiple pictures
were taken over time of the same particle. This allowed for another 3-D plot to
be made that showed the movement of the particle over time.
In Figure~\ref{fig:3D}, one hundred reconstructions were done, and then they
were composed into a three-dimensional image by stacking all 100 images on top
of one another. The images were taken \SI{100}{\micro\second} apart. 
In this case, there was approximately a 1 micron movement 
of the particle along the $z$-axis, so instead of plotting this small movement,
the $z$-axis was set to be time instead, so that it could be seen how far the
particle moved in the $x$ or $y$-axis over time.
%in the $z$-axis, so instead time was used so that it could be seen how far the
%particle moved in the $x$-axis  or $y$-axis over time. 
This allowed for the
particle to be tracked in all three dimensions and in time. This is a
capability not possible with optical bright-field microscopy.

%the reconstruction revealed more detail in the object
%than the in-focus holographic image did. This was because of the high contrast that was
%obtained in the reconstruction, using the hologram subtracted from the
%background. 
%The high contrast allowed for the small details to be seen, which were
%approximately one to two microns wide.

\begin{figure}[ht!]
    \begin{center}

        \begin{subfigure}[t]{0.4\textwidth}
            \label{fig:147focus}
            \includegraphics[width=\textwidth]{pictures/147focus.png}
            \caption{The in-focus holographic image that had little contrast.}
        \end{subfigure}
        \hspace*{\fill}
        \begin{subfigure}[t]{0.4\textwidth}
            \label{fig:148holo}
            \includegraphics[width=\textwidth]{pictures/148holo.png}
            \caption{A hologram of the in-focus image.}
        \end{subfigure}
        \\
        \begin{subfigure}[t]{\textwidth}
            \label{fig:148rec}
            \includegraphics[width=\textwidth]{pictures/148rec.png}
            \caption{A reconstruction at a $z$ value of
                \SI{44.898}{\micro\meter}.}
        \end{subfigure}


    \end{center}
    \caption{%
        The reconstruction of the hologram showed more details in the object
        than the in-focus image did. Although the fringes seemed to be interfering with
        one another, the resolution was still high allowing for one to
        two micron sized details to be observed.
%        The higher contrast of the reconstruction
%        allowed for these details to be seen.
    }%
    \label{fig:148}
\end{figure}
\clearpage

\begin{figure}[ht!]
    \begin{center}

        \includegraphics[width=\textwidth]{pictures/particlexyz.png}
%        \begin{subfigure}[t]{0.4\textwidth}
%            \label{fig:particlexz}
%            \includegraphics[width=\textwidth]{particlexz}
%            \caption{A paricle shown in the x vs z direction}
%        \end{subfigure}
%        \hspace*{\fill}
%        \begin{subfigure}[t]{0.4\textwidth}
%            \label{fig:particleyz}
%            \includegraphics[width=\textwidth]{particleyz}
%            \caption{A paricle shown in the z vs y direction}
%        \end{subfigure}
%        \\
%        \begin{subfigure}[t]{0.6\textwidth}
%            \label{fig:particlexy}
%            \includegraphics[width=\textwidth]{particlexy}
%            \caption{A paricle shown in the xy-plane.}
%        \end{subfigure}


    \end{center}
    \caption{%
        A 3-dimensional image of a particle, with the axes in
        pixels. This particle appears cylindrical due to the residual intensity
        from the hologram.
    }%
    \label{fig:3Dparticle}
\end{figure}

\begin{figure}[ht!]
    \begin{center}

        \begin{subfigure}[t]{0.4\textwidth}
            \label{fig:3Dx}
            \includegraphics[width=\textwidth]{x}
            \caption{Movement along the $x$-axis over time.}
        \end{subfigure}
        \hspace*{\fill}
        \begin{subfigure}[t]{0.4\textwidth}
            \label{fig:3Dy}
            \includegraphics[width=\textwidth]{y}
            \caption{Movement along the $y$-axis over time.}
        \end{subfigure}
        \\
        \begin{subfigure}[t]{0.6\textwidth}
            \label{fig:3Dyx}
            \includegraphics[width=\textwidth]{yx}
            \caption{Movement along the $xy$-plane.}
        \end{subfigure}


    \end{center}
    \caption{%
        A 3-dimensional image of a particle that had moved by about 1 micron
        along the $z$-axis over 100 time slices at \SI{100}{\micro\second} for
        each time slice.
%        A 3-dimensional image of a particle that had little movement in the
%        $z$-axis, so the $z$-axis was replaced with time to show movement over
%        time.
    }%
    \label{fig:3D}
\end{figure}

%Since holograms contain information of the 3-D structure of the sample, a 3-D
%plot of particle movement or depth profiling within the specimen can be
%performed. In Figure~\ref{fig:3Dparticle}, a particle was reconstructed using over
%twenty different $z$ values for a total range of 30 microns. The particle was then ploted in 3-D
%space. This particle was moving when it came into view, so multiple pictures
%were taken over time of the same particle. This allowed for another 3-D plot to
%be made that showed the movement of the particle over time.
%In Figure~\ref{fig:3D}, one hundred reconstructions were done, and then they
%were composed into a three-dimensional image by stacking all 100 images on top
%of one another. The images were taken \SI{100}{\micro\second} apart. 
%In this case, there was approximately a 1 micron movement 
%of the particle along the $z$-axis, so instead of plotting this small movement,
%the $z$-axis was set to be time instead, so that it could be seen how far the
%particle moved in the $x$ or $y$-axis over time.
%%in the $z$-axis, so instead time was used so that it could be seen how far the
%%particle moved in the $x$-axis  or $y$-axis over time. 
%This allowed for the
%particle to be tracked in all three dimensions and in time. This is a
%capability not possible with optical bright-field microscopy.

%All of the reconstructions have some fine details shown which can be used to
%estimate the resolution of the Manoharan Lab setup, which Figure~\ref{fig:148} gave at
%least a 2 micron resolution since two micron objects can be seen in the
%numerical reconstruction of the hologram.

%All of the reconstructions calculated from the Harvard University images had a pixel
%spacing of 0.17 microns and the image size was 1024 by 1024 pixels.
%Therefore each image is \SI{174.08}{\micro\meter} by \SI{174.08}{\micro\meter}.

%\bibliographystyle{plain}
%\bibliography{HonoursBiblio/HonoursBiblio.bib}
%\end{document}
