%
%---------------------------------------------------------
% Header: This file includes all the packages and custom definitions
% we have used in this example thesis.
%----------------------------------------------------------

\documentclass[12pt,twoside,openright]{report}

%---------------------
% START: Packages
%---------------------
\usepackage{textcomp}
\usepackage[latin1]{inputenc}
\usepackage{amsmath}
\usepackage{amsfonts}
\usepackage{amssymb}
\usepackage{amsthm}
\usepackage{graphicx}
\usepackage{soul}
\usepackage{listings}
%\usepackage{subfig}
\usepackage{verbatim}
\usepackage{alltt}
\usepackage{siunitx}
\usepackage{color}

\usepackage{caption}
\usepackage{subcaption}
\usepackage{lmodern}

% There are used in the graphics chapter.
% You can delete the following two lines if you use no tikz/pgf graphics
% in your thesis.
\usepackage{tikz}
\usepackage{pgfplots}

% Load the natbib citation package: set the citations to be numerical
% with square brackets separated by commas.
\usepackage[numbers,square,comma]{natbib}

% Now include the Acadia thesis style
\usepackage{acadia-hon-thesis}

% Load the hyperref package.
% The options tell it to (a) use hyper-links to pages with Roman
% numerals that are different than pages with Arabic numbers, and
% (b) tell Adobe reader to show a page number matching the thesis page
% number (rather than sequentially numbering the PDF pages from 1).
\usepackage[plainpages=false,pdfpagelabels]{hyperref}

% The information in the first three lines here goes into the PDF
% document properties.
% The rest of the lines define options related to hyper-links.
% colorlinks: typeset links in the given colours
%	      (otherwise an ugly box is drawn around the links, although
%	       it is only seen on the screen, not in printed copies)
% A newer option (since May 2011) would be to just use the hidelinks option.
% Note: pdfprintscaling=None should discourage Adobe reader from wanting to
% scale your pages to fit printable area when you print from Adobe reader.
\hypersetup{%
    pdftitle={Digital In-line Holography and the Numerical Reconstruction of Holograms},
    pdfauthor={Wesley A. Bowman},
    pdfkeywords={holography,microscopy,in-line,hologram},
    colorlinks = true,
    linkcolor = black,
    anchorcolor = black,
    citecolor = black,
    filecolor = black,
    urlcolor = black,
    pdfprintscaling=None
}


% Load the algorithm/mic packages and use chapter-wise numbering
\usepackage[chapter]{algorithm}
\usepackage{algorithmic}



%---------------------
% END: Packages
%---------------------
\bibliographystyle{plainnat}

% The depth of the table of contents: change the MAXIMUM depth of
% citations in your table of contents.
\setcounter{tocdepth}{6}

%
% Some definitions of commands used in this thesis
%

% For instance, if you have an acronym you like to use, then define a
% command, it's faster and if the acronym changes you only have to
% change it in one place.
\def\sysacro{SPECIALACRONYM}

% Allow us to change the margins easily and at will
\newenvironment{changemargin}[2]{%
  \begin{list}{}{%
    \setlength{\topsep}{0pt}%
    \setlength{\leftmargin}{#1}%
    \setlength{\rightmargin}{#2}%
    \setlength{\listparindent}{\parindent}%
    \setlength{\itemindent}{\parindent}%
    \setlength{\parsep}{\parskip}%
  }%
  \item[]}{\end{list}}

%setup the default format of listings
\lstset{%
    basicstyle=\footnotesize,
    numbers=left,
    xleftmargin=5mm,
    linewidth=\textwidth,
    breaklines,
    frame=tb,
    frameround=fttt
}

\definecolor{dkgreen}{rgb}{0,0.6,0}
\definecolor{gray}{rgb}{0.5,0.5,0.5}
\definecolor{mauve}{rgb}{0.58,0,0.82}

% A new definition style
\newtheoremstyle{defstyle}	% name
    {3pt}			% Space above
    {3pt}			% Space below
    {}				% Body font
    {}				% Indent amount
    {\itshape}			% Theorem head font
    {:}				% Punctuation after theorem head
    {.5em}			% Space after theorem head
    {}		% Theorem head spec (can be left empty,meaning 'normal�)
\theoremstyle{definition}
\newtheorem{definition}{Definition}[chapter]


% Change comment style for algorithms
\renewcommand{\algorithmiccomment}[1]{/*#1*/}
% Change Require: to Input: for algorithms
\renewcommand{\algorithmicrequire}{\textbf{Input:}}
% Change Ensure: to Output: for algorithms
\renewcommand{\algorithmicensure}{\textbf{Output:}}

\newcommand{\BigO}[1]{\ensuremath{\operatorname{O}\bigl(#1\bigr)}}

\newfloat{Code}{H}{myc}

%\begin{document}
%
%\chapter{Results} \label{chap:Results}

\chapter{Results} \label{chap:Results}
\section{Harvard University}

HoloPy was the open-source Python module used for all of the reconstructions
done. When this module was found, the opportunity was presented to travel to
Harvard University and meet with the developer to get some incite on how to
build an efficient microscope and to learn how to properly use HoloPy. While at
Harvard University, the Manoharan Lab graciously allowed for holograms to be
taken using their setup so that there could be a comparison to the microscope
that was to be built.

\begin{figure}[ht!]
    \begin{center}

        \begin{subfigure}[t]{0.4\textwidth}
            \label{fig:second}
            \includegraphics[width=\textwidth]{pictures/20130814_141000.jpg}
            \caption{The Harvard setup has both bright field and holographic
            capabilities.}
        \end{subfigure}
            %
                \hspace*{\fill}
    %
        \begin{subfigure}[t]{0.4\textwidth}
            \label{fig:third}
            \includegraphics[width=\textwidth]{pictures/20130814_141007.jpg}
            \caption{The monitor allows for real time viewing of the images
            before and after acquisition.}
        \end{subfigure}
        \\

        \begin{subfigure}[t]{0.4\textwidth}
            \label{fig:fourth}
            \includegraphics[width=\textwidth]{pictures/20130814_141151.jpg}
            \caption{The sample with the laser beam going through it.}
        \end{subfigure}
        \hspace*{\fill}
        \begin{subfigure}[t]{0.4\textwidth}%[Caption of Second Figure]{%
            \label{fig:fifth}
            \includegraphics[width=\textwidth]{pictures/20130814_141310.jpg}
            \caption{The laser setup before it gets to the sample.}
        \end{subfigure}

    \end{center}
    \caption{%
        The Harvard University equipment setup.
    }%
    \label{fig:harvardSetup}
\end{figure}

A lot of the images taken at Harvard University have objects in them which were
unknown, since the sample was used was water from the Charles River. Not
having known what the objects were did not cause an issue though, since
reconstruction was the only purpose of looking at the objects.


\begin{figure}
    \begin{center}
        \includegraphics[width=\textwidth]{pictures/128focus.png}
        \caption{The in-focus image for Figures~\ref{fig:129}-\ref{fig:131}. }
        \label{fig:128focus}
    \end{center}
\end{figure}





\begin{figure}[ht!]
    \begin{center}

            %
                %\hspace*{\fill}
    %
        \begin{subfigure}[t]{0.4\textwidth}
            \label{fig:129holo}
            \includegraphics[width=\textwidth]{pictures/129holo.png}
            \caption{Hologram}
        \end{subfigure}
        \\
        \begin{subfigure}[t]{0.4\textwidth}
            \label{fig:129rec1}
            \includegraphics[width=\textwidth]{pictures/129rec1.png}
            \caption{Reconstruction at one $z$ value.}
        \end{subfigure}
            %
                \hspace*{\fill}
    %
%
        \begin{subfigure}[t]{0.4\textwidth}
            \label{fig:129rec2}
            \includegraphics[width=\textwidth]{pictures/129rec2.png}
            \caption{Reconstruction at a different $z$ value.}
        \end{subfigure}


    \end{center}
    \caption{%
        This hologram has the least amount of fridges in these series of
        holograms.
    }%
    \label{fig:129}
\end{figure}


\begin{figure}[ht!]
    \begin{center}

        \begin{subfigure}[t]{0.4\textwidth}
            \label{fig:130holo}
            \includegraphics[width=\textwidth]{pictures/130holo.png}
            \caption{Hologram}
        \end{subfigure}
        \\
        \begin{subfigure}[t]{0.4\textwidth}
            \label{fig:130rec1}
            \includegraphics[width=\textwidth]{pictures/130rec1.png}
            \caption{Reconstruction at one value of $z$.}
        \end{subfigure}
            %
                \hspace*{\fill}
    %
%
        \begin{subfigure}[t]{0.4\textwidth}
            \label{fig:130rec2}
            \includegraphics[width=\textwidth]{pictures/130rec2.png}
            \caption{Reconstruction at a different value of $z$.}
        \end{subfigure}


    \end{center}
    \caption{%
        This hologram is an in-between hologram in terms of the fridges.
    }%
    \label{fig:130}
\end{figure}


\begin{figure}[ht!]
    \begin{center}

        \begin{subfigure}[t]{0.4\textwidth}
            \label{fig:131holo}
            \includegraphics[width=\textwidth]{pictures/131holo.png}
            \caption{Hologram}
        \end{subfigure}
        \\
        \begin{subfigure}[t]{\textwidth}
            \label{fig:131rec}
            \includegraphics[width=\textwidth]{pictures/131rec.png}
            \caption{Reconstruction}
        \end{subfigure}


    \end{center}
    \caption{%
        This hologram has the most fridges and seems to comtain the least
        amount of detail, but when reconstructed, it is seen that more detail
        than the previous reconstructions can be obtained, while also losing
        different details.
    }%
    \label{fig:131}
\end{figure}

The images shown below show a few different
techniques of holography that is advantageous compared to bright field optics.
For Figures~\ref{fig:128focus}-\ref{fig:131}, the difference between the amount of fridges in the hologram
are shown, along with reconstructions at different $z$ values. The advantages
in holography are that the images do not need to be in-focus to be viewed with
good resolution. This advantage leads to the other advantage that objects do
not have to lie in the same plane to be viewed.

\begin{figure}[ht!]
    \begin{center}

        \begin{subfigure}[t]{0.4\textwidth}
            \label{fig:132focus}
            \includegraphics[width=\textwidth]{pictures/132focus.png}
            \caption{In-focus image}
        \end{subfigure}
            %
                %\hspace*{\fill}
    %
        \begin{subfigure}[t]{0.4\textwidth}
            \label{fig:133holo}
            \includegraphics[width=\textwidth]{pictures/133holo.png}
            \caption{Hologram of the in-focus image.}
        \end{subfigure}
        \\
        \begin{subfigure}[t]{\textwidth}
            \label{fig:133rec}
            \includegraphics[width=\textwidth]{pictures/133rec.png}
            \caption{Reconstruction of the hologram.}
        \end{subfigure}


    \end{center}
    \caption{%
       It can be seen that the reconstruction is highly detailed compared to
       the low level of detail in the hologram.
    }%
    \label{fig:133}
\end{figure}

A typical reconstruction is shown in Figure~\ref{fig:133}. Even with some
interference from objects, the reconstruction for the rectangular object is
still fairly detailed and distinguished.


\begin{figure}
\begin{center}
    \includegraphics[width=\textwidth]{pictures/136focus.png}
    \caption{The in-focus image for Figures~\ref{fig:139} and~\ref{fig:140}. }
    \label{fig:136focus}
\end{center}
\end{figure}


\begin{figure}[ht!]
    \begin{center}

        \begin{subfigure}[t]{0.4\textwidth}
            \label{fig:139holo}
            \includegraphics[width=\textwidth]{pictures/139holo.png}
            \caption{Hologram}
        \end{subfigure}
        \\
        \begin{subfigure}[t]{\textwidth}
            \label{fig:139rec}
            \includegraphics[width=\textwidth]{pictures/139rec.png}
            \caption{Reconstruction}
        \end{subfigure}


    \end{center}
    \caption{%
        Hologram that has a fair level of detail, and then reconstructed.
    }%
    \label{fig:139}
\end{figure}


\begin{figure}[ht!]
    \begin{center}

        \begin{subfigure}[t]{0.4\textwidth}
            \label{fig:140holo}
            \includegraphics[width=\textwidth]{pictures/140holo.png}
            \caption{Hologram}
        \end{subfigure}
        \\
        \begin{subfigure}[t]{\textwidth}
            \label{fig:140rec}
            \includegraphics[width=\textwidth]{pictures/140rec.png}
            \caption{Reconstruction}
        \end{subfigure}


    \end{center}
    \caption{%
        The same object with more fridges and less detail that provides a
        better detail reconstruction than Figure~\ref{fig:139}
    }%
    \label{fig:140}
\end{figure}

Once again, in Figures~\ref{fig:136focus}-\ref{fig:140}, the difference in the
amount of fridges is shown. In this case, the more fridges led to a higher
resolution inside the object.


\begin{figure}[ht!]
    \begin{center}

        \begin{subfigure}[t]{0.4\textwidth}
            \label{fig:147focus}
            \includegraphics[width=\textwidth]{pictures/147focus.png}
            \caption{In-focus image}
        \end{subfigure}
        \begin{subfigure}[t]{0.4\textwidth}
            \label{fig:148holo}
            \includegraphics[width=\textwidth]{pictures/148holo.png}
            \caption{Hologram}
        \end{subfigure}
        \\
        \begin{subfigure}[t]{\textwidth}
            \label{fig:148rec}
            \includegraphics[width=\textwidth]{pictures/148rec.png}
            \caption{Reconstruction}
        \end{subfigure}


    \end{center}
    \caption{%
        The reconstruction shows more detail than the in-focus image.
    }%
    \label{fig:148}
\end{figure}

In Figures~\ref{fig:148}, there is more detail in the reconstruction than there
is in the in-focus image. This is because of the high contrast that is obtained
when doing a reconstruction, allowing the details to be more easily seen.

All of the reconstructions have some fine details shown. For all of the
reconstructions done from the Harvard University images, the spacing in-between
the pixels was 0.17 microns, and the image size was 1024 by 1024 pixels.









%\end{document}
