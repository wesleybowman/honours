%
%---------------------------------------------------------
% Header: This file includes all the packages and custom definitions
% we have used in this example thesis.
%----------------------------------------------------------

\documentclass[12pt,twoside,openright]{report}

%---------------------
% START: Packages
%---------------------
\usepackage{textcomp}
\usepackage[latin1]{inputenc}
\usepackage{amsmath}
\usepackage{amsfonts}
\usepackage{amssymb}
\usepackage{amsthm}
\usepackage{graphicx}
\usepackage{soul}
\usepackage{listings}
%\usepackage{subfig}
\usepackage{verbatim}
\usepackage{alltt}
\usepackage{siunitx}
\usepackage{color}

\usepackage{caption}
\usepackage{subcaption}
\usepackage{lmodern}

% There are used in the graphics chapter.
% You can delete the following two lines if you use no tikz/pgf graphics
% in your thesis.
\usepackage{tikz}
\usepackage{pgfplots}

% Load the natbib citation package: set the citations to be numerical
% with square brackets separated by commas.
\usepackage[numbers,square,comma]{natbib}

% Now include the Acadia thesis style
\usepackage{acadia-hon-thesis}

% Load the hyperref package.
% The options tell it to (a) use hyper-links to pages with Roman
% numerals that are different than pages with Arabic numbers, and
% (b) tell Adobe reader to show a page number matching the thesis page
% number (rather than sequentially numbering the PDF pages from 1).
\usepackage[plainpages=false,pdfpagelabels]{hyperref}

% The information in the first three lines here goes into the PDF
% document properties.
% The rest of the lines define options related to hyper-links.
% colorlinks: typeset links in the given colours
%	      (otherwise an ugly box is drawn around the links, although
%	       it is only seen on the screen, not in printed copies)
% A newer option (since May 2011) would be to just use the hidelinks option.
% Note: pdfprintscaling=None should discourage Adobe reader from wanting to
% scale your pages to fit printable area when you print from Adobe reader.
\hypersetup{%
    pdftitle={Digital In-line Holography and the Numerical Reconstruction of Holograms},
    pdfauthor={Wesley A. Bowman},
    pdfkeywords={holography,microscopy,in-line,hologram},
    colorlinks = true,
    linkcolor = black,
    anchorcolor = black,
    citecolor = black,
    filecolor = black,
    urlcolor = black,
    pdfprintscaling=None
}


% Load the algorithm/mic packages and use chapter-wise numbering
\usepackage[chapter]{algorithm}
\usepackage{algorithmic}



%---------------------
% END: Packages
%---------------------
\bibliographystyle{plainnat}

% The depth of the table of contents: change the MAXIMUM depth of
% citations in your table of contents.
\setcounter{tocdepth}{6}

%
% Some definitions of commands used in this thesis
%

% For instance, if you have an acronym you like to use, then define a
% command, it's faster and if the acronym changes you only have to
% change it in one place.
\def\sysacro{SPECIALACRONYM}

% Allow us to change the margins easily and at will
\newenvironment{changemargin}[2]{%
  \begin{list}{}{%
    \setlength{\topsep}{0pt}%
    \setlength{\leftmargin}{#1}%
    \setlength{\rightmargin}{#2}%
    \setlength{\listparindent}{\parindent}%
    \setlength{\itemindent}{\parindent}%
    \setlength{\parsep}{\parskip}%
  }%
  \item[]}{\end{list}}

%setup the default format of listings
\lstset{%
    basicstyle=\footnotesize,
    numbers=left,
    xleftmargin=5mm,
    linewidth=\textwidth,
    breaklines,
    frame=tb,
    frameround=fttt
}

\definecolor{dkgreen}{rgb}{0,0.6,0}
\definecolor{gray}{rgb}{0.5,0.5,0.5}
\definecolor{mauve}{rgb}{0.58,0,0.82}

% A new definition style
\newtheoremstyle{defstyle}	% name
    {3pt}			% Space above
    {3pt}			% Space below
    {}				% Body font
    {}				% Indent amount
    {\itshape}			% Theorem head font
    {:}				% Punctuation after theorem head
    {.5em}			% Space after theorem head
    {}		% Theorem head spec (can be left empty,meaning 'normal�)
\theoremstyle{definition}
\newtheorem{definition}{Definition}[chapter]


% Change comment style for algorithms
\renewcommand{\algorithmiccomment}[1]{/*#1*/}
% Change Require: to Input: for algorithms
\renewcommand{\algorithmicrequire}{\textbf{Input:}}
% Change Ensure: to Output: for algorithms
\renewcommand{\algorithmicensure}{\textbf{Output:}}

\newcommand{\BigO}[1]{\ensuremath{\operatorname{O}\bigl(#1\bigr)}}

\newfloat{Code}{H}{myc}

%\begin{document}

\chapter{Appendix} \label{chap:Appendix}

\section{Fourier Transform}

In science, one of the most fundamental principles is the superposition
principle. The superposition principle tells us that for a linear system
composed of different functions is the sum of the responses that would have
been caused by the different functions individually. Green's function is
written as the
superposition of infinitely many impulse response functions, which then makes
the response a superposition of impulse responses. Fourier transforms are like
Green's function, but instead of using impulse response functions, it uses
sinusoids instead.

Think of sinusoids (sines and cosines) as being the fundamental building blocks
for functions. From these functions, we can write any other function as a
superposition of those two functions, whether they be periodic or not. This can
be represented with an infinite series, known as the Fourier series.

First, let's suppose that $f(t)$ is a periodic function with a period
$T$, so that
\begin{equation}
    f(t)=f(t+T)
\end{equation}
for all values of $t$ \cite{DanielSteck}. We can represent this as

\begin{equation}
    f(t) = a_0 + b_0 + 2\sum_{n=1}^{\infty} a_n \cos(n\omega
    t)+2\sum_{n=1}^{\infty}b_n\sin(n\omega t)
    \label{eq:fourierseries}
\end{equation}
where $\omega=2\pi\nu=\frac{2\pi}{T}$ is the angular frequency, and $\nu$ is the frequency.
We can define the complex Fourier coefficient as

\begin{equation}
    c_n \equiv a_n+ib_n
\end{equation}
which allows for Equation~\ref{eq:fourierseries} to be rewritten as

\begin{equation}
    f(t) = a_0 + b_0 +\sum_{n=1}^{\infty} a_n \left( e^{in\omega t} +
    e^{-in\omega t} \right) - i\sum_{n=1}^{\infty} b_n \left( e^{in\omega t} -
    e^{-in\omega t} \right)
\end{equation}
This can be simplified farther

\begin{equation}
    f(t) = \sum_{n=-\infty}^{\infty}a_ne^{-in\omega t} + i\sum_{n=-\infty}^{\infty}b_ne^{-in\omega t}
\end{equation}

Now, using the complex Fourier coefficient defined earlier, we can get the
complex Fourier series

\begin{equation}
    f(t)=\sum_{n=-\infty}^{\infty}c_ne^{-in\omega t}
\end{equation}

Now, these same harmonic functions can be used to build aperiodic functions. An
aperiodic function corresponds to $T\rightarrow \infty$, so we need a
continuous spectrum to represent an aperiodic function, since there is more
information in the function than the aperiodic case.

\begin{equation}
    f(t)=\sum_{n=-\infty}^{\infty}c_ne^{-in\omega t} \rightarrow
    f(t)=\frac{1}{2\pi}\int_{-\infty}^{\infty} \tilde{f}(\omega) e^{-i\omega t}~d\omega
\end{equation}
This is the inverse Fourier transform. To find the amplitudes, $c_n$, which are
also called the Fourier transform,

\begin{equation}
    c_n = \frac{1}{T}\int_0^Te^{in\omega t} f(t) dt \rightarrow
    \tilde{f}(\omega) =\int_{-\infty}^{\infty}f(t)e^{i\omega t}~dt
\end{equation}
In a more common notation,

\begin{equation}
    \mathfrak{F} =\int_{-\infty}^{\infty}f(t)e^{i\omega t}~dt
    \qquad
    \mathfrak{F}^{-1} =\int_{-\infty}^{\infty}f(\omega)e^{-i\omega t}~d\omega
\end{equation}
where $\mathfrak{F}$ is the Fourier transform of  $f(t)$ and
$\mathfrak{F}^{-1}$ is the inverse Fourier transform of $f(t)$.


%*=============================================================================================*

