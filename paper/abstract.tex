\documentclass[12pt]{article}

\usepackage{amsthm}
\usepackage{amssymb}
\usepackage{amsmath}
\usepackage{tikz}
\usepackage{epsfig}
\usepackage{enumerate}
\usepackage{graphicx}
\usepackage{tabularx}
\usepackage{siunitx}
%%\usepackage[margin=0.5in]{geometry}

\usepackage{url}
\urlstyle{same}

%FlowChart Stuff
\usetikzlibrary{shapes,arrows}
\tikzstyle{decision} = [diamond, draw, text width=4.5em, text badly centered, node distance=3cm, inner sep=0pt]
\tikzstyle{block} = [rectangle, draw, text width=5em, text centered, rounded corners, minimum height=4em]

%%theorems and stuff
\newtheorem{theorem}{Theorem}[section]
\newtheorem{prop}[theorem]{Proposition}
\newtheorem{lemma}[theorem]{Lemma}
\newtheorem{result}[theorem]{Result}
\newtheorem{definition}[theorem]{Definition}

%%definitions
\theoremstyle{definition}
\newtheorem{example}{Example}[section]

%tab command
\newcommand{\tab}{\hspace*{2em}}

%%degree command
\newcommand{\degree}{\ensuremath{^\circ}}

%%Begin%%

\begin{document}

\title{Digital In-Line Holography and the Numerical Reconstruction of Holograms}
\author{Wesley Bowman}
\date{\today}
\maketitle

\begin{abstract}

    A hologram is the recorded interference pattern between a reference
    wave and a scattered wave from an object. Although, the recording medium only
    captures
    the intensity of the light from these waves, the information about both
    the amplitude and the phase is still present in this recording, due to the
    interference fringes produced by the superposition of these two waves.
    This information can be reconstructed numerically using the
    Fresnel-Kirchoff integral, which can be transformed into a
    numerically efficient calculation using the convolution theorem.

    The setup used in this thesis was in-line holography.
    In Digital In-line Holographic Microscopy (DIHM), all of the components 
    are aligned in a linear setup, and 
    both the reference and scattered waves are
    recorded on the same detector. The setup used in this thesis contained a
    laser, two objective lenses, the object holder, a neutral density (ND) filter, and
    a charged coupled device (CCD). The CCD allowed for direct 
    digital recording of the holograms.

    Once the holograms were obtained from the DIHM, an open-source Python
    library, HoloPy, was used to reconstruct the holographic images. Images
    from both Harvard and Acadia Universities were recorded and reconstructed, which
    demonstrated the successful
    acquisition of holograms and the reconstruction method.
    The DIHM that was constructed at Acadia achieved a resolution of approximately
    \SI{2.7}{\micro\meter}.


\end{abstract}

\end{document}
