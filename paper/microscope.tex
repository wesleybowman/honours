%
%---------------------------------------------------------
% Header: This file includes all the packages and custom definitions
% we have used in this example thesis.
%----------------------------------------------------------

\documentclass[12pt,twoside,openright]{report}

%---------------------
% START: Packages
%---------------------
\usepackage{textcomp}
\usepackage[latin1]{inputenc}
\usepackage{amsmath}
\usepackage{amsfonts}
\usepackage{amssymb}
\usepackage{amsthm}
\usepackage{graphicx}
\usepackage{soul}
\usepackage{listings}
%\usepackage{subfig}
\usepackage{verbatim}
\usepackage{alltt}
\usepackage{siunitx}
\usepackage{color}

\usepackage{caption}
\usepackage{subcaption}
\usepackage{lmodern}

% There are used in the graphics chapter.
% You can delete the following two lines if you use no tikz/pgf graphics
% in your thesis.
\usepackage{tikz}
\usepackage{pgfplots}

% Load the natbib citation package: set the citations to be numerical
% with square brackets separated by commas.
\usepackage[numbers,square,comma]{natbib}

% Now include the Acadia thesis style
\usepackage{acadia-hon-thesis}

% Load the hyperref package.
% The options tell it to (a) use hyper-links to pages with Roman
% numerals that are different than pages with Arabic numbers, and
% (b) tell Adobe reader to show a page number matching the thesis page
% number (rather than sequentially numbering the PDF pages from 1).
\usepackage[plainpages=false,pdfpagelabels]{hyperref}

% The information in the first three lines here goes into the PDF
% document properties.
% The rest of the lines define options related to hyper-links.
% colorlinks: typeset links in the given colours
%	      (otherwise an ugly box is drawn around the links, although
%	       it is only seen on the screen, not in printed copies)
% A newer option (since May 2011) would be to just use the hidelinks option.
% Note: pdfprintscaling=None should discourage Adobe reader from wanting to
% scale your pages to fit printable area when you print from Adobe reader.
\hypersetup{%
    pdftitle={Digital In-line Holography and the Numerical Reconstruction of Holograms},
    pdfauthor={Wesley A. Bowman},
    pdfkeywords={holography,microscopy,in-line,hologram},
    colorlinks = true,
    linkcolor = black,
    anchorcolor = black,
    citecolor = black,
    filecolor = black,
    urlcolor = black,
    pdfprintscaling=None
}


% Load the algorithm/mic packages and use chapter-wise numbering
\usepackage[chapter]{algorithm}
\usepackage{algorithmic}



%---------------------
% END: Packages
%---------------------
\bibliographystyle{plainnat}

% The depth of the table of contents: change the MAXIMUM depth of
% citations in your table of contents.
\setcounter{tocdepth}{6}

%
% Some definitions of commands used in this thesis
%

% For instance, if you have an acronym you like to use, then define a
% command, it's faster and if the acronym changes you only have to
% change it in one place.
\def\sysacro{SPECIALACRONYM}

% Allow us to change the margins easily and at will
\newenvironment{changemargin}[2]{%
  \begin{list}{}{%
    \setlength{\topsep}{0pt}%
    \setlength{\leftmargin}{#1}%
    \setlength{\rightmargin}{#2}%
    \setlength{\listparindent}{\parindent}%
    \setlength{\itemindent}{\parindent}%
    \setlength{\parsep}{\parskip}%
  }%
  \item[]}{\end{list}}

%setup the default format of listings
\lstset{%
    basicstyle=\footnotesize,
    numbers=left,
    xleftmargin=5mm,
    linewidth=\textwidth,
    breaklines,
    frame=tb,
    frameround=fttt
}

\definecolor{dkgreen}{rgb}{0,0.6,0}
\definecolor{gray}{rgb}{0.5,0.5,0.5}
\definecolor{mauve}{rgb}{0.58,0,0.82}

% A new definition style
\newtheoremstyle{defstyle}	% name
    {3pt}			% Space above
    {3pt}			% Space below
    {}				% Body font
    {}				% Indent amount
    {\itshape}			% Theorem head font
    {:}				% Punctuation after theorem head
    {.5em}			% Space after theorem head
    {}		% Theorem head spec (can be left empty,meaning 'normal�)
\theoremstyle{definition}
\newtheorem{definition}{Definition}[chapter]


% Change comment style for algorithms
\renewcommand{\algorithmiccomment}[1]{/*#1*/}
% Change Require: to Input: for algorithms
\renewcommand{\algorithmicrequire}{\textbf{Input:}}
% Change Ensure: to Output: for algorithms
\renewcommand{\algorithmicensure}{\textbf{Output:}}

\newcommand{\BigO}[1]{\ensuremath{\operatorname{O}\bigl(#1\bigr)}}

\newfloat{Code}{H}{myc}

%\begin{document}

\chapter{Holographic Microscopy Apparatus} \label{chap:Micro}

%One of the main goals of this research was to construct a microscope that was
%able to create holograms that could be reconstructed numerically and the 
%technique implemented was in-line, plane-wave
%holography.

%All parts used where placed on a
%optical rail that was horizontal.

%\section{Pinhole Holograms}
%This was the first microscope that was tried in the lab. This microscope
%consisted of a laser, a pinhole, the object to be reconstructed, and a charged
%coupled device, or a CCD.
%
%A 632 nm laser was mostly used, but a 450 nm laser was tested at one point as
%well to see if it helped the resolution and intensity. The pinhole needed to be
%small, so a \SI{10}{\micro\meter}, \SI{5}{\micro\meter}, \SI{1}{\micro\meter} pinhole was used. The smaller the
%aperture, the higher the resolution could be. The samples were nothing in
%particular for this stage, as it was a testing stage to see if reconstructions
%could be done with the setup, therefore hairs or fibre optic cables were used
%for the sample. The CCD used was a webcam, since it was an effective CCD for
%the price, and since this was still in the testing process it didn't feel
%necessary to go overboard with an expensive CCD.
%
%The problem with this setup was that it lead to a point source illumination
%which lead to spherical propagation of light after the aperture. This created
%a problem in the reconstruction, since the reconstruction methods that were
%being used were for plane-wave holography. Another problem was that to do
%pinhole holography, you have to be very precise with the laser hitting the
%pinhole, and then very precise with the rest of the equipment placing after the
%pinhole as well. The camera needs to get very close to the object, and the
%object needs to be close to the pinhole. The amount of precision needed was
%just not possible with the equipment that we had available.
%

%\section{Plane-Wave Holography}

A simple form of holography that was tolerant to imperfections in the
experimental setup, plane-wave holography was used in this work. This
setup consisted of a laser, two objective lenses, the sample, and a CCD\@. There was
also a neutral density filter (ND filter) that could be inserted after the
laser to lower the
%which allowed for the
intensity of light and
%to be adjusted so as not to 
prevent saturation of the CCD\@.
The optical density, or absorbance, of a ND filter is defined as,

\begin{equation}
    d = -\log_{10}\frac{I}{I_0},
    \label{eq:ND}
\end{equation}
where $d$ is the optical density, $I_0$ is the initial intensity, and $I$ is
the intensity after the ND filter.
A schematic diagram of
the experimental setup and the actual setup used are shown in Figures~\ref{fig:inlineSetup}
and~\ref{fig:OurMicroscope}, respectively.



\begin{figure}[htbp!]
    \begin{center}
        \includegraphics[scale=0.5]{inlineSetup}
    \end{center}
    \caption{A diagram of the in-line, plane-wave holography microscope.}
    \label{fig:inlineSetup}
\end{figure}


\begin{figure}[ht!]
    \begin{center}
        \begin{subfigure}[t]{\textwidth}
            \label{fig:OurMicroscopeSide}
            \includegraphics[width=\textwidth]{pictures/microscopesideFIXED.png}
            \caption{}
%            \caption{(i.) HeNe laser (ii.) ND filter (iii.) 10x objective (iv.)
%                slide holder (v.) 20x objective (vi.) CCD
%            }
        \end{subfigure}
            %
%        \\
%    %
%        \begin{subfigure}[t]{\textwidth}
%            \label{fig:OurMicroscopeSideUp}
%            \includegraphics[width=\textwidth]{pictures/microscopesideupFIXED.png}
%            \caption{}
%        \end{subfigure}
%        \\
%
%        \begin{subfigure}[t]{\textwidth}
%            \label{fig:OurMicroscopeUp}
%            \includegraphics[width=\textwidth]{pictures/microscopeupFIXED.png}
%            \caption{}
%        \end{subfigure}

    \end{center}
%    \caption{%
%        The actual setup used as seen from (a) side view, (b) a different angle
%        that allows for a better view of the CCD, the ND filter, and the
%        objectives, and (c) a top-down view that shows both optical rails.
%    }%
    %\label{fig:OurMicroscope}
\end{figure}

\begin{figure}[ht!]
    \ContinuedFloat
    \begin{center}
        \begin{subfigure}[t]{\textwidth}
            \label{fig:OurMicroscopeSideUp}
            \includegraphics[width=\textwidth]{pictures/microscopesideupFIXED.png}
            \caption{}
%            \caption{(i.) HeNe laser (ii.) ND filter (iii.) 10x objective (iv.)
%                slide holder (v.) 20x objective (vi.) CCD
%            }
        \end{subfigure}

    \end{center}
    %\label{fig:OurMicroscope}
\end{figure}



\begin{figure}[ht!]
    \ContinuedFloat
    \begin{center}

        \begin{subfigure}[t]{\textwidth}
            \label{fig:OurMicroscopeUp}
            \includegraphics[width=\textwidth]{pictures/microscopeupFIXED.png}
%            \caption{
%                \begin{enumerate}
%                    \item HeNe laser
%                    \item ND filter
%                    \item 10x objective
%                    \item slide holder
%                    \item 20x objective
%                    \item CCD
%                \end{enumerate}
%            }
            \caption{}
%            \caption{(i.) HeNe laser (ii.) ND filter (iii.) 10x objective (iv.)
%                slide holder (v.) 20x objective (vi.) CCD
%            }
        \end{subfigure}

    \end{center}
    \caption{%
        The experimental setup as viewed from (a) the side, (b)
        an angle, and (c) the top. The parts are
            (i.) HeNe laser, (ii.) ND filter, (iii.) 10x objective, (iv.)
                slide holder, (v.) 20x objective, and (vi.) CCD.
    }%
    \label{fig:OurMicroscope}
\end{figure}


%\begin{figure}
%\begin{center}
%    \includegraphics[scale=0.10]{OurMicroscope}
%\end{center}
%\caption{Our actual setup. Need a new updated and better picture of this.}
%\label{fig:OurMicroscope}
%\end{figure}

%\begin{figure}[htbp!]
%\begin{center}
%    \includegraphics[scale=0.05]{pictures/microscopeside.jpg}
%\end{center}
%\caption{Our actual setup as seen from the side view.}
%\label{fig:OurMicroscopeSide}
%\end{figure}
%
%\begin{figure}[htbp!]
%\begin{center}
%    \includegraphics[scale=0.05]{pictures/microscopesideup.jpg}
%\end{center}
%\caption{A different angle of our setup that allows for a view of the most of
%    the optical parts, including the CCD.}
%\label{fig:OurMicroscopeSideUp}
%\end{figure}
%
%\begin{figure}[htbp!]
%\begin{center}
%    \includegraphics[scale=0.05]{pictures/microscopeup.jpg}
%\end{center}
%\caption{A top-down view of the setup that shows both optical rails.}
%\label{fig:OurMicroscopeUp}
%\end{figure}
%
\clearpage


A 632 nm, 0.5 mW helium-neon laser was used as the coherent light
source.
%The same
%lasers used in the pinhole setup were used here. 
The CCD was a Panasonic Lumix DMC-GF3 interchangeable lens camera.
Interchangeable lens cameras have an important advantage over normal digital
cameras in that they have no mirrors or prisms in front of
the CCD\@, which allowed for direct exposure of the incoming light with minimal
aberrations.
%the Panasonic Lumix DMC-GF3 camera body only to
%be bought, since there was no need for a lens. The CCD was directly
%exposed to the incoming light, which allowed for minimal aberrations.

The Panasonic Lumix DMC-GF3 camera also allowed for high shutter speeds, low ISO
%(which means low noise),
and had approximately 12 million pixels.
The two objective lenses were used
before and after the sample in the setup. The first objective, which was a 10x
objective lens, was to %there for
culminate the light from the laser and would not be needed if the laser
had a collimated beam of appropriate diameter. The purpose of the second
objective lens, which was a 20x
objective lens, was to
magnify the resulting interference patterns created
when the light passed through the sample. This allowed for the holograms to be
recorded onto the CCD using the full area of the CCD.
%, as it allowed the CCD to be fully used.
Both of the objective lenses and the object were placed on
three-axis mounts so that they could be adjusted and aligned accurately.
%The Edmund optics 
The particular neutral density filter used depended on the sample opacity to avoid
saturation of the CCD.
%that was being viewed, and if the CCD was being saturated. Saturating the CCD does not allow for a good hologram reconstruction
%since some of the information is missing.
%There were three different ND filters with
The three different optical density values used were 1.0, 0.6,
and a 0.3, allowing for the intensity of the light from the laser to be
adjusted over a wide range.
%The ND filter went after the laser
%and before the first objective. It was placed here for ease of use and access,
%but could be moved to anywhere in the setup.


%\end{document}
