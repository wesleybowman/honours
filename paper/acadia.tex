%
%---------------------------------------------------------
% Header: This file includes all the packages and custom definitions
% we have used in this example thesis.
%----------------------------------------------------------

\documentclass[12pt,twoside,openright]{report}

%---------------------
% START: Packages
%---------------------
\usepackage{textcomp}
\usepackage[latin1]{inputenc}
\usepackage{amsmath}
\usepackage{amsfonts}
\usepackage{amssymb}
\usepackage{amsthm}
\usepackage{graphicx}
\usepackage{soul}
\usepackage{listings}
%\usepackage{subfig}
\usepackage{verbatim}
\usepackage{alltt}
\usepackage{siunitx}
\usepackage{color}

\usepackage{caption}
\usepackage{subcaption}
\usepackage{lmodern}

% There are used in the graphics chapter.
% You can delete the following two lines if you use no tikz/pgf graphics
% in your thesis.
\usepackage{tikz}
\usepackage{pgfplots}

% Load the natbib citation package: set the citations to be numerical
% with square brackets separated by commas.
\usepackage[numbers,square,comma]{natbib}

% Now include the Acadia thesis style
\usepackage{acadia-hon-thesis}

% Load the hyperref package.
% The options tell it to (a) use hyper-links to pages with Roman
% numerals that are different than pages with Arabic numbers, and
% (b) tell Adobe reader to show a page number matching the thesis page
% number (rather than sequentially numbering the PDF pages from 1).
\usepackage[plainpages=false,pdfpagelabels]{hyperref}

% The information in the first three lines here goes into the PDF
% document properties.
% The rest of the lines define options related to hyper-links.
% colorlinks: typeset links in the given colours
%	      (otherwise an ugly box is drawn around the links, although
%	       it is only seen on the screen, not in printed copies)
% A newer option (since May 2011) would be to just use the hidelinks option.
% Note: pdfprintscaling=None should discourage Adobe reader from wanting to
% scale your pages to fit printable area when you print from Adobe reader.
\hypersetup{%
    pdftitle={Digital In-line Holography and the Numerical Reconstruction of Holograms},
    pdfauthor={Wesley A. Bowman},
    pdfkeywords={holography,microscopy,in-line,hologram},
    colorlinks = true,
    linkcolor = black,
    anchorcolor = black,
    citecolor = black,
    filecolor = black,
    urlcolor = black,
    pdfprintscaling=None
}


% Load the algorithm/mic packages and use chapter-wise numbering
\usepackage[chapter]{algorithm}
\usepackage{algorithmic}



%---------------------
% END: Packages
%---------------------
\bibliographystyle{plainnat}

% The depth of the table of contents: change the MAXIMUM depth of
% citations in your table of contents.
\setcounter{tocdepth}{6}

%
% Some definitions of commands used in this thesis
%

% For instance, if you have an acronym you like to use, then define a
% command, it's faster and if the acronym changes you only have to
% change it in one place.
\def\sysacro{SPECIALACRONYM}

% Allow us to change the margins easily and at will
\newenvironment{changemargin}[2]{%
  \begin{list}{}{%
    \setlength{\topsep}{0pt}%
    \setlength{\leftmargin}{#1}%
    \setlength{\rightmargin}{#2}%
    \setlength{\listparindent}{\parindent}%
    \setlength{\itemindent}{\parindent}%
    \setlength{\parsep}{\parskip}%
  }%
  \item[]}{\end{list}}

%setup the default format of listings
\lstset{%
    basicstyle=\footnotesize,
    numbers=left,
    xleftmargin=5mm,
    linewidth=\textwidth,
    breaklines,
    frame=tb,
    frameround=fttt
}

\definecolor{dkgreen}{rgb}{0,0.6,0}
\definecolor{gray}{rgb}{0.5,0.5,0.5}
\definecolor{mauve}{rgb}{0.58,0,0.82}

% A new definition style
\newtheoremstyle{defstyle}	% name
    {3pt}			% Space above
    {3pt}			% Space below
    {}				% Body font
    {}				% Indent amount
    {\itshape}			% Theorem head font
    {:}				% Punctuation after theorem head
    {.5em}			% Space after theorem head
    {}		% Theorem head spec (can be left empty,meaning 'normal�)
\theoremstyle{definition}
\newtheorem{definition}{Definition}[chapter]


% Change comment style for algorithms
\renewcommand{\algorithmiccomment}[1]{/*#1*/}
% Change Require: to Input: for algorithms
\renewcommand{\algorithmicrequire}{\textbf{Input:}}
% Change Ensure: to Output: for algorithms
\renewcommand{\algorithmicensure}{\textbf{Output:}}

\newcommand{\BigO}[1]{\ensuremath{\operatorname{O}\bigl(#1\bigr)}}

\newfloat{Code}{H}{myc}

%\begin{document}
%\chapter{Results} \label{chap:Results}

\section{Acadia University}

%Once the microscope was constructed, images could be taken in the lab. Many
%biological objects were tried, but due to the amount of objects in the
%three-dimensional space, there became too many fridges to resolve into a
%reconstruction. The reconstructions could be done, but no details could be
%seen.

A series of holograms were recorded using the apparatus displayed in
Figure~\ref{fig:OurMicroscope}.

As with the images obtained from Harvard University, some of the objects shown are
objects that appeared in the slide without intention but made for interesting
holograms and reconstructions. With the CCD that was used in the microscope,
the spacing per pixel was about twice that of the Harvard images at
approximately 0.34
microns. The image size was 4000 by 3000 pixels, and hence the images were
\SI{1360}{\micro\meter} by \SI{1020}{\micro\meter} in the $x$ and $y$
dimensions, respectively. The values of $z$ used for
reconstructions in the Acadia images are larger since the objects were farther
away from the CCD than in the Harvard images.

%In Figure~\ref{fig:68_74} the object was a fibre from the lens tissue used to
The object in Figure~\ref{fig:68_74} was a fibre from the lens tissue used to
clean the slide. This fibre was found in the hologram and then after the
hologram was captured the slide was repositioned to take the in-focus
holographic image.
This demonstrates an advantage of holography over optical bright-field
microscopy. In a bright-field microscope this object might have been
overlooked since the fibre was not at the focal length of the objective lens. Holography allowed for
this object to be found and reconstructed without the need to find the focal
plane of the object. 
%This fibre also had a lot of fringes that were spaced
%appropriately so that a detailed reconstruction could still be achieved. 
%The fringe spacing is high because the object was so far away from the CCD, which
%is why the reconstruction could still be done with the object being 1.26 cm
%away from the CCD.

Figure~\ref{fig:72_74} was captured while trying to view a solution of
colloidal diamond, but the object, which was probably dust, was imaged. This made
for an interesting reconstruction since the object had a loop in it. Even
though there are interfering fringes at the loop, which was thought might cause
a loss of resolution and make the space between the object
unrecognizable, the loop was still able to be resolved. The largest spacing
between two sides of the loop was approximately \SI{23}{\micro\meter}.
%, which
%gave confidence that the resolution of the microscope was at least
%\SI{23}{\micro\meter}.

Figures~\ref{fig:187_200} and~\ref{fig:198_200} were images of colloidal diamond
with a particle diameter of about \SI{5}{\micro\meter}.
In Figure~\ref{fig:187_200}, the
solution used was highly populated and fringes overlapped with one another.
This was not ideal as overlapping fringes can cause a reduction in
resolution. However, the 5 micron particles were still clearly resolved.
%Figure~\ref{fig:198_200} was 
A more diluted solution of colloidal diamond that
had a population more ideally suited for recording holograms is shown in Figure~\ref{fig:198_200}.
Some of the
particles are very close to one another and yet they were still resolved
into two distinct 5 micron particles, as shown at spot $i$. 
A line scan of these two particles was performed and a fit was made with two
%shown at spot $i.$, a fit can be made with two
Gaussian distributions. In Figure~\ref{fig:linescan}, the two
Gaussians and their sum fit along with the original data gathered from Figure~\ref{fig:198_200}
(c) are presented. Using the individual fits of the Gaussians, along with their sum, the resolution could be
determined by moving one Gaussian toward the other until two distinct Gaussians
could no longer be recognized. This is known as the Rayleigh
criterion. 
%which states that 
When the full-width half-maximum (FWHM) of two Gaussians
are directly overlapping, then this defines the minimum resolvable separation
between the two Gaussians. Equation \ref{eq:Rayleigh} is the mathematical
formulation of the Rayleigh criterion,

\begin{equation}
    \delta_{Rayleigh} \propto \frac{\lambda}{NA}
    \label{eq:Rayleigh}
\end{equation}
%
where $\delta$ is the minimum resolvable separation, $\lambda$ is the
wavelength of the laser, and $NA$ is the numerical aperture of the optical
system \cite{Goodman}.
The separation of the FWHM on each of the individual Gaussians was
approximately 5 microns which corresponded closely to the known particle size and
showed that the individual fits were able to reliably model the particles in this
scenario.


In Figure~\ref{fig:linescan} (c), the Rayleigh criterion is shown, with
$\delta= ~$\SI{2.72}{\micro\meter}. Once the FWHM of each of the Gaussians move past one
another, as displayed in Figure~\ref{fig:linescan} (d), the sum of the
Gaussians can no longer be displayed as two distinct particles. The $NA$ of the
objective lens used in Figure~\ref{fig:OurMicroscope} was 0.25
and using
%, which allowed us to use 
Equation~\ref{eq:Rayleigh} to determine the theoretical minimum
separation of two particles that could still be resolvable,
\begin{equation}
    \delta_{Rayleigh} \propto \frac{635~\mathrm{nm}}{0.25} = 2540~\mathrm{nm} =
    2.54~\mathrm{\SI{}{\micro\meter}} ,
\end{equation}
which is close to the value obtained in Figure~\ref{fig:linescan} (c).
This gives confidence in the results obtained and indicates that the particles
can be resolved up to a lateral
geometrical distance apart of approximately
\SI{2.5}{\micro\meter}.


\begin{figure}[ht!]
    \begin{center}

        \begin{subfigure}[t]{0.4\textwidth}
            \label{fig:69_focus}
            \includegraphics[width=\textwidth]{pictures/69_focus.png}
            \caption{The in-focus holographic image of a wood fibre.}
        \end{subfigure}
            %
                \hspace*{\fill}
    %
        \begin{subfigure}[t]{0.4\textwidth}
            \label{fig:68_74holo}
            \includegraphics[width=\textwidth]{pictures/68_74holo.png}
            \caption{A hologram of the in-focus image.}
        \end{subfigure}
        \\
        \begin{subfigure}[t]{\textwidth}
            \label{fig:68_74rec}
            \includegraphics[width=\textwidth]{pictures/68_74rec.png}
            \caption{A reconstruction at a $z$ value of
                \SI{12580}{\micro\meter}.}
        \end{subfigure}


    \end{center}
    \caption{%
        The reconstruction of a wood fibre, which was likely from a lens tissue
        during cleaning of the slide, obtained from a hologram that had
        approximately six fringes next to the object. The object was far away from
        the CCD.
    }%
    \label{fig:68_74}
\end{figure}



%\begin{figure}[ht!]
%    \begin{center}
%
%        \begin{subfigure}[t]{0.4\textwidth}
%            \label{fig:86_focus}
%            \includegraphics[width=\textwidth]{pictures/86_focus.png}
%            \caption{In-focus image}
%        \end{subfigure}
%            %
%                %\hspace*{\fill}
%    %
%        \begin{subfigure}[t]{0.4\textwidth}
%            \label{fig:85_88holo}
%            \includegraphics[width=\textwidth]{pictures/85_88holo.png}
%            \caption{Hologram}
%        \end{subfigure}
%        \\
%        \begin{subfigure}[t]{\textwidth}
%            \label{fig:85_88rec}
%            \includegraphics[width=\textwidth]{pictures/85_88rec.png}
%            \caption{Reconstruction}
%        \end{subfigure}
%
%
%    \end{center}
%    \caption{%
%        An image that shows the kind of resolution that can be achieved with
%        holography and our microscope.
%    }%
%    \label{fig:85_88}
%\end{figure}


\begin{figure}[ht!]
    \begin{center}

            %
                %\hspace*{\fill}
    %
        \begin{subfigure}[t]{0.4\textwidth}
            \label{fig:72_74holo}
            \includegraphics[width=\textwidth]{pictures/72_74holo.png}
            \caption{A hologram of a piece of dust in solution.}
        \end{subfigure}
        \\
        \begin{subfigure}[t]{\textwidth}
            \label{fig:72_74rec}
            \includegraphics[width=\textwidth]{pictures/72_74rec.png}
            \caption{A holographic reconstruction at a $z$ value of
                \SI{1706}{\micro\meter}.}
        \end{subfigure}


    \end{center}
    \caption{%
        A piece of dust that in some places was \SI{5}{\micro\meter} in
        thickness. Even with overlapping of the interference fringes, the loop in the
        dust was still sharply resolved.
    }%
    \label{fig:72_74}
\end{figure}


\clearpage


\begin{figure}[ht!]
    \begin{center}

            %
                %\hspace*{\fill}
    %
        \begin{subfigure}[t]{0.4\textwidth}
            \label{fig:187_200holo}
            \includegraphics[width=\textwidth]{pictures/187_200holo.png}
            \caption{A hologram of colloidal diamond.}
        \end{subfigure}
        \\
        \begin{subfigure}[t]{\textwidth}
            \label{fig:187_200rec}
            \includegraphics[width=\textwidth]{pictures/187_200rec.png}
            \caption{A holographic reconstruction at a $z$ value of \SI{675}{\micro\meter}.}
        \end{subfigure}


    \end{center}
    \caption{%
        This was an example of a highly populated hologram. While this is not
        an ideal situation since too many fringes can result in a loss of
        resolution, the colloidal diamond particles could still be
        resolved.
    }%
    \label{fig:187_200}
\end{figure}

\clearpage


\begin{figure}[ht!]
    \begin{center}

        \begin{subfigure}[t]{0.4\textwidth}
            \label{fig:196_focus}
            \includegraphics[width=\textwidth]{pictures/196_focus.png}
            \caption{An in-focus holographic image of colloidal diamond.}
        \end{subfigure}
            %
                \hspace*{\fill}
    %
        \begin{subfigure}[t]{0.4\textwidth}
            \label{fig:198_200holo}
            \includegraphics[width=\textwidth]{pictures/198_200holo.png}
            \caption{A hologram of the in-focus colloidal diamond.}
        \end{subfigure}
        \\
        \begin{subfigure}[t]{\textwidth}
            \label{fig:198_200rec}
            \includegraphics[width=\textwidth]{pictures/198_200rec.png}
            \caption{A reconstruction at a $z$ value of \SI{500}{\micro\meter}.}
        \end{subfigure}


    \end{center}
    \caption{%
        This hologram was recorded from a more diluted solution than was used
        in
        Figure~\ref{fig:187_200}.
        The \SI{5}{\micro\meter} size of the
        particles was resolved in this reconstruction, even when the particles
        were close together.
    }%
    \label{fig:198_200}
\end{figure}

%\clearpage


%Figure~\ref{fig:72_74} was captured while trying to view a solution of
%colloidal diamond, but the object, which was probably dust, was imaged. This made
%for an interesting reconstruction since the object had a loop in it. Even
%though there are interfering fringes at the loop, which was thought might cause
%a loss of resolution and make the space between the object
%unrecognizable, the loop was still able to be resolved. The largest spacing
%between two sides of the loop was approximately \SI{23}{\micro\meter}.
%, which
%gave confidence that the resolution of the microscope was at least
%\SI{23}{\micro\meter}.

%Figures~\ref{fig:187_200} and~\ref{fig:198_200} were images of colloidal diamond
%with a particle diameter of about \SI{5}{\micro\meter}.
%In Figure~\ref{fig:187_200}, the
%solution used was highly populated and fringes overlapped with one another.
%This was not ideal as overlapping fringes can cause a reduction in
%resolution. However, the 5 micron particles were still clearly resolved.
%%Figure~\ref{fig:198_200} was 
%A more diluted solution of colloidal diamond that
%had a population more ideally suited for recording holograms is shown in Figure~\ref{fig:198_200}.
%Some of the
%particles are very close to one another and yet they were still resolved
%into two distinct 5 micron particles, as shown at spot $i$. 
%A line scan of these two particles was performed and a fit was made with two
%%shown at spot $i.$, a fit can be made with two
%Gaussian distributions. In Figure~\ref{fig:linescan}, the two
%Gaussians and their sum fit along with the original data gathered from Figure~\ref{fig:198_200}
%(c) are presented. Using the individual fits of the Gaussians, along with their sum, the resolution could be
%determined by moving one Gaussian toward the other until two distinct Gaussians
%could no longer be recognized. This is known as the Rayleigh
%criterion. 
%%which states that 
%When the full-width half-maximum (FWHM) of two Gaussians
%are directly overlapping, then this defines the minimum resolvable separation
%between the two Gaussians. Equation \ref{eq:Rayleigh} is the mathematical
%formulation of the Rayleigh criterion,
%
%\begin{equation}
%    \delta_{Rayleigh} \propto \frac{\lambda}{NA}
%    \label{eq:Rayleigh}
%\end{equation}
%%
%where $\delta$ is the minimum resolvable separation, $\lambda$ is the
%wavelength of the laser, and $NA$ is the numerical aperture of the optical
%system \cite{Goodman}.
%The separation of the FWHM on each of the individual Gaussians was
%approximately 5 microns which corresponded closely to the known particle size and
%showed that the individual fits were able to reliably model the particles in this
%scenario.


\begin{figure}[ht!]
    \begin{center}

        \begin{subfigure}[t]{\textwidth}
            \label{fig:linescan_original}
            \includegraphics[width=\textwidth]{pictures/resolutionOriginal.png}
            \caption{A Gaussian curve was fit to each peak in the original linescan.
                The fit was used as an approximation to the two particles shown
            at spot $i$ in Figure~\ref{fig:198_200}.}
        \end{subfigure}
            %
        \\
                %\hspace*{\fill}
    %
        \begin{subfigure}[t]{\textwidth}
            \label{fig:linescan_new1}
            \includegraphics[width=\textwidth]{pictures/resolutionNew1.png}
            \caption{The red Gaussian curve was moved closer toward the blue
            Gaussian. The distinctness of the peaks gets weaker as the
        ``particles'' are moved closer to one another.}
        \end{subfigure}

    \end{center}
\end{figure}

\begin{figure}[ht!]
    \ContinuedFloat
    \begin{center}
        \begin{subfigure}[t]{\textwidth}
            \label{fig:linescan_rayleigh}
            \includegraphics[width=\textwidth]{pictures/resolutionRayleigh.png}
            \caption{When the full-width half-maximum (FWHM) of the fits are
            overlayed, the Rayleigh criterion is met and the two point-sources are
            ``barely resolved.'' \cite{Goodman}}
        \end{subfigure}
        \\
        \begin{subfigure}[t]{\textwidth}
            \label{fig:linescan_new2}
            \includegraphics[width=\textwidth]{pictures/resolutionNew2.png}
            \caption{Once the FWHM values have completely overlapped the two
            particles are no longer distinguishable.}
        \end{subfigure}

    \end{center}
    \caption{%
        It is displayed in (c) that we can still resolve both of the
        \SI{5}{\micro\meter} particles when $\delta$, the distance between the
        peaks, is
        \SI{2.72}{\micro\meter}.
    }%
    \label{fig:linescan}

\end{figure}

%In Figure~\ref{fig:linescan} (c), the Rayleigh criterion is shown, with
%$\delta=$\SI{2.72}{\micro\meter}. Once the FWHM of each of the Gaussians move past one
%another, as displayed in Figure~\ref{fig:linescan} (d), the sum of the
%Gaussians can no longer be displayed as two distinct particles. The NA of the
%objective lens used in Figure~\ref{fig:OurMicroscope} was 0.25
%and using
%%, which allowed us to use 
%Equation~\ref{eq:Rayleigh} to determine the theoretical minimum
%separation of two particles that could still be resolvable,
%\begin{equation}
%    \delta_{Rayleigh} \propto \frac{635~\mathrm{nm}}{0.25} = 2540~\mathrm{nm} =
%    2.54~\mathrm{\SI{}{\micro\meter}} ,
%\end{equation}
%which is close to the value obtained in Figure~\ref{fig:linescan} (c).
%This gives confidence in the results obtained and indicates that the particles
%can be resolved up to a lateral
%geometrical distance apart of approximately
%\SI{2.5}{\micro\meter}.






%\bibliographystyle{plain}
%\bibliography{HonoursBiblio/HonoursBiblio.bib}
%\end{document}
