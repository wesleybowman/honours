%
%---------------------------------------------------------
% Header: This file includes all the packages and custom definitions
% we have used in this example thesis.
%----------------------------------------------------------

\documentclass[12pt,twoside,openright]{report}

%---------------------
% START: Packages
%---------------------
\usepackage{textcomp}
\usepackage[latin1]{inputenc}
\usepackage{amsmath}
\usepackage{amsfonts}
\usepackage{amssymb}
\usepackage{amsthm}
\usepackage{graphicx}
\usepackage{soul}
\usepackage{listings}
%\usepackage{subfig}
\usepackage{verbatim}
\usepackage{alltt}
\usepackage{siunitx}
\usepackage{color}

\usepackage{caption}
\usepackage{subcaption}
\usepackage{lmodern}

% There are used in the graphics chapter.
% You can delete the following two lines if you use no tikz/pgf graphics
% in your thesis.
\usepackage{tikz}
\usepackage{pgfplots}

% Load the natbib citation package: set the citations to be numerical
% with square brackets separated by commas.
\usepackage[numbers,square,comma]{natbib}

% Now include the Acadia thesis style
\usepackage{acadia-hon-thesis}

% Load the hyperref package.
% The options tell it to (a) use hyper-links to pages with Roman
% numerals that are different than pages with Arabic numbers, and
% (b) tell Adobe reader to show a page number matching the thesis page
% number (rather than sequentially numbering the PDF pages from 1).
\usepackage[plainpages=false,pdfpagelabels]{hyperref}

% The information in the first three lines here goes into the PDF
% document properties.
% The rest of the lines define options related to hyper-links.
% colorlinks: typeset links in the given colours
%	      (otherwise an ugly box is drawn around the links, although
%	       it is only seen on the screen, not in printed copies)
% A newer option (since May 2011) would be to just use the hidelinks option.
% Note: pdfprintscaling=None should discourage Adobe reader from wanting to
% scale your pages to fit printable area when you print from Adobe reader.
\hypersetup{%
    pdftitle={Digital In-line Holography and the Numerical Reconstruction of Holograms},
    pdfauthor={Wesley A. Bowman},
    pdfkeywords={holography,microscopy,in-line,hologram},
    colorlinks = true,
    linkcolor = black,
    anchorcolor = black,
    citecolor = black,
    filecolor = black,
    urlcolor = black,
    pdfprintscaling=None
}


% Load the algorithm/mic packages and use chapter-wise numbering
\usepackage[chapter]{algorithm}
\usepackage{algorithmic}



%---------------------
% END: Packages
%---------------------
\bibliographystyle{plainnat}

% The depth of the table of contents: change the MAXIMUM depth of
% citations in your table of contents.
\setcounter{tocdepth}{6}

%
% Some definitions of commands used in this thesis
%

% For instance, if you have an acronym you like to use, then define a
% command, it's faster and if the acronym changes you only have to
% change it in one place.
\def\sysacro{SPECIALACRONYM}

% Allow us to change the margins easily and at will
\newenvironment{changemargin}[2]{%
  \begin{list}{}{%
    \setlength{\topsep}{0pt}%
    \setlength{\leftmargin}{#1}%
    \setlength{\rightmargin}{#2}%
    \setlength{\listparindent}{\parindent}%
    \setlength{\itemindent}{\parindent}%
    \setlength{\parsep}{\parskip}%
  }%
  \item[]}{\end{list}}

%setup the default format of listings
\lstset{%
    basicstyle=\footnotesize,
    numbers=left,
    xleftmargin=5mm,
    linewidth=\textwidth,
    breaklines,
    frame=tb,
    frameround=fttt
}

\definecolor{dkgreen}{rgb}{0,0.6,0}
\definecolor{gray}{rgb}{0.5,0.5,0.5}
\definecolor{mauve}{rgb}{0.58,0,0.82}

% A new definition style
\newtheoremstyle{defstyle}	% name
    {3pt}			% Space above
    {3pt}			% Space below
    {}				% Body font
    {}				% Indent amount
    {\itshape}			% Theorem head font
    {:}				% Punctuation after theorem head
    {.5em}			% Space after theorem head
    {}		% Theorem head spec (can be left empty,meaning 'normal�)
\theoremstyle{definition}
\newtheorem{definition}{Definition}[chapter]


% Change comment style for algorithms
\renewcommand{\algorithmiccomment}[1]{/*#1*/}
% Change Require: to Input: for algorithms
\renewcommand{\algorithmicrequire}{\textbf{Input:}}
% Change Ensure: to Output: for algorithms
\renewcommand{\algorithmicensure}{\textbf{Output:}}

\newcommand{\BigO}[1]{\ensuremath{\operatorname{O}\bigl(#1\bigr)}}

\newfloat{Code}{H}{myc}

%\begin{document}
%\chapter{Results} \label{chap:Results}

\section{Acadia University}

Once the microscope was constructed, images could be taken in the lab. Many
biological objects were tried, but due to the amount of objects in the
three-dimensional space, there became too many fridges to resolve into a
reconstruction. The reconstructions could be done, but no details could be
seen.

Like the images obtained from Harvard University, some of the objects shown are
objects that appeared in the slide without intention, but made for interesting
holograms and reconstructions. With the CCD that was used in the microscope,
the spacing per pixel was about twice that of the Harvard images at around 0.34
microns. The image size was 4000 by 3000 pixels.

\begin{figure}[ht!]
    \begin{center}

        \begin{subfigure}[t]{0.4\textwidth}
            \label{fig:69_focus}
            \includegraphics[width=\textwidth]{pictures/69_focus.png}
            \caption{In-focus image}
        \end{subfigure}
            %
                %\hspace*{\fill}
    %
        \begin{subfigure}[t]{0.4\textwidth}
            \label{fig:68_74holo}
            \includegraphics[width=\textwidth]{pictures/68_74holo.png}
            \caption{Hologram}
        \end{subfigure}
        \\
        \begin{subfigure}[t]{\textwidth}
            \label{fig:68_74rec}
            \includegraphics[width=\textwidth]{pictures/68_74rec.png}
            \caption{Reconstruction}
        \end{subfigure}


    \end{center}
    \caption{%
        The reconstruction, which was obtained from a low detail hologram,
        matches the in-focus image.
    }%
    \label{fig:68_74}
\end{figure}



\begin{figure}[ht!]
    \begin{center}

        \begin{subfigure}[t]{0.4\textwidth}
            \label{fig:86_focus}
            \includegraphics[width=\textwidth]{pictures/86_focus.png}
            \caption{In-focus image}
        \end{subfigure}
            %
                %\hspace*{\fill}
    %
        \begin{subfigure}[t]{0.4\textwidth}
            \label{fig:85_88holo}
            \includegraphics[width=\textwidth]{pictures/85_88holo.png}
            \caption{Hologram}
        \end{subfigure}
        \\
        \begin{subfigure}[t]{\textwidth}
            \label{fig:85_88rec}
            \includegraphics[width=\textwidth]{pictures/85_88rec.png}
            \caption{Reconstruction}
        \end{subfigure}


    \end{center}
    \caption{%
        An image that shows the kind of resolution that can be achieved with
        holography and our microscope.
    }%
    \label{fig:85_88}
\end{figure}


\begin{figure}[ht!]
    \begin{center}

            %
                %\hspace*{\fill}
    %
        \begin{subfigure}[t]{0.4\textwidth}
            \label{fig:72_74holo}
            \includegraphics[width=\textwidth]{pictures/72_74holo.png}
            \caption{Hologram}
        \end{subfigure}
        \\
        \begin{subfigure}[t]{\textwidth}
            \label{fig:72_74rec}
            \includegraphics[width=\textwidth]{pictures/72_74rec.png}
            \caption{Reconstruction}
        \end{subfigure}


    \end{center}
    \caption{%
        Another object that has a high resolution for it's small size.
    }%
    \label{fig:72_74}
\end{figure}


\clearpage


\begin{figure}[ht!]
    \begin{center}

            %
                %\hspace*{\fill}
    %
        \begin{subfigure}[t]{0.4\textwidth}
            \label{fig:187_200holo}
            \includegraphics[width=\textwidth]{pictures/187_200holo.png}
            \caption{Hologram}
        \end{subfigure}
        \\
        \begin{subfigure}[t]{\textwidth}
            \label{fig:187_200rec}
            \includegraphics[width=\textwidth]{pictures/187_200rec.png}
            \caption{Reconstruction}
        \end{subfigure}


    \end{center}
    \caption{%
        This hologram was highly populated, and a good reconstruction was still
        obtained, even with the high amount of fridges from separate objects.
    }%
    \label{fig:187_200}
\end{figure}

\clearpage


\begin{figure}[ht!]
    \begin{center}

        \begin{subfigure}[t]{0.4\textwidth}
            \label{fig:196_focus}
            \includegraphics[width=\textwidth]{pictures/196_focus.png}
            \caption{In-focus image}
        \end{subfigure}
            %
                %\hspace*{\fill}
    %
        \begin{subfigure}[t]{0.4\textwidth}
            \label{fig:198_200holo}
            \includegraphics[width=\textwidth]{pictures/198_200holo.png}
            \caption{Hologram}
        \end{subfigure}
        \\
        \begin{subfigure}[t]{\textwidth}
            \label{fig:198_200rec}
            \includegraphics[width=\textwidth]{pictures/198_200rec.png}
            \caption{Reconstruction}
        \end{subfigure}


    \end{center}
    \caption{%
        This hologram was a less densely populated version of
        Figure~\ref{fig:187_200}.
    }%
    \label{fig:198_200}
\end{figure}

\clearpage

All of the images taken in the lab at Acadia University have a high resolution.
This resolution was obtained with a minimum setup, and saving resources when
they could be spared.
Figures~\ref{fig:85_88},~\ref{fig:187_200},~\ref{fig:198_200} display how small
the pixel resolution gets. In these images, the object is colloidal silica
which has a diameter of around 5 microns.

In Figure~\ref{fig:72_74}, the object is very small, and the level of detail is
still high. This image is interesting because of the way it overlaps on itself,
and the fridges overlap, and yet the reconstruction has a high resolution.
In Figures~\ref{fig:187_200} and~\ref{fig:198_200}, the difference between two
different dilutions of the same solution are shown. In this instance, it didn't
matter, but one could see that if the densely populated hologram got too much
more populated, the fridges could case a lose of resolution, and when the
objects are 5 microns, this could result in objects missing, or a
reconstruction so overcrowded that objects could not be distinguished from one
another.








%\end{document}
